%\renewcommand{\theequation}{\theenumi}
%\begin{enumerate}[label=\arabic*.,ref=\thesection.\theenumi]
%\numberwithin{equation}{enumi}
% \renewcommand{\thefigure}{\theenumi}
% \numberwithin{figure}{enumi}
%
\item Draw Fig. \ref{fig:tri_right_angle} for $a = 4, c =3$.
\label{const:tri_right_angle}
%
\begin{figure}[!ht]
\centering
\resizebox{\columnwidth}{!}{\input{./figs/triangle/tri_right_angle.tex}}
\caption{Right Angled Triangle}
\label{fig:tri_right_angle}	
\end{figure}
\\
\solution The vertices of $\triangle ABC$ are 
\begin{align}
\vec{A} = \myvec{0\\c} = \myvec{0\\3}, \vec{B} = \myvec{0\\0}, \vec{C} = \myvec{a\\0}=\myvec{4\\0}
\end{align}
%
The python code for  Fig. \ref{fig:tri_right_angle} is
\begin{lstlisting}
codes/triangle/tri_right_angle.py
\end{lstlisting}
%
and the equivalent latex-tikz code is
%
\begin{lstlisting}
figs/triangle/tri_right_angle.tex
\end{lstlisting}
%
The above latex code can be compiled as a standalone document as
%
\begin{lstlisting}
figs/triangle/tri_right_angle_alone.tex
\end{lstlisting}
%

\item Draw Fig. \ref{fig:tri_polar} for $a = 4, c =3$.
\label{const:tri_polar}
%
\\
\solution 
 The vertex  $\vec{A}$ can  be expressed  in {\em polar coordinate form} as
%\label{prob:tri_polar}
%
\begin{align}
\vec{A} = b\myvec{\cos \theta\\  \sin \theta} 
\end{align}
%
where
\begin{align}
b = \sqrt{a^2+c^2} = 5, \tan \theta = \frac{3}{4}
\end{align}
%The vertices of $\triangle ABC$ are 
%\begin{align}
%\vec{A} = \myvec{a\\c} = \myvec{4\\3}, \vec{B} = \myvec{a\\0}  = \myvec{4\\0}, \vec{C} = \myvec{0\\0}.
%\end{align}
%
The python code for  Fig. \ref{fig:tri_polar} is
\begin{lstlisting}
codes/triangle/tri_polar.py
\end{lstlisting}
%
and the equivalent latex-tikz code is
%
\begin{lstlisting}
figs/triangle/tri_polar.tex
\end{lstlisting}
\begin{figure}[!ht]
\centering
\resizebox{\columnwidth}{!}{\input{./figs/triangle/tri_polar.tex}}
\caption{Right Angled Triangle}
\label{fig:tri_polar}	
\end{figure}
%
\item Draw Fig. \ref{fig:tri_sss} with $a=6$, $b=5$  and $c=4$.  
\label{const:tri_sss}
\begin{figure}[!ht]
	\begin{center}
			\resizebox{\columnwidth}{!}{\input{./figs/triangle/tri_sss.tex}}
	\end{center}
	\caption{}
	\label{fig:tri_sss}	
\end{figure}
\\
\solution Let the vertices of  $\triangle ABC$ and $\vec{D}$ be 
\begin{align}
\label{eq:tri_basic}
\vec{A} = \myvec{p\\q}, \vec{B} = \myvec{0\\0}, \vec{C} = \myvec{a\\0}, \vec{D} = \myvec{p\\0}
\end{align}
%

Then
\begin{align}
\label{eq:c_tricoord}
AB &= \norm{\vec{A}-\vec{B}}^2 = \norm{\vec{A}}^2  = c^2 \quad \because \vec{B} = \vec{0}
\\
\label{eq:a_tricoord}
BC &= \norm{\vec{C}-\vec{B}}^2 = \norm{\vec{C}}^2  = a^2
\\
AC &= \norm{\vec{A}-\vec{C}}^2 =    b^2
\label{eq:b_tricoord}
\end{align}
%
From \eqref{eq:b_tricoord},
\begin{align}
b^2 &=\norm{\vec{A}-\vec{C}}^2 = \norm{\vec{A}-\vec{C}}^T\norm{\vec{A}-\vec{C}}  
\\
&= \vec{A}^T\vec{A}+\vec{C}^T\vec{C}-\vec{A}^T\vec{C} - \vec{C}^T\vec{A} 
\\
&= \norm{\vec{A}}^2 + \norm{\vec{C}}^2 - 2\vec{A}^T\vec{C} \quad \brak{\because \vec{A}^T\vec{C} = \vec{C}^T\vec{A} } 
\label{eq:tri_const_norm_ac}
\\
&= a^2+c^2-2ap
\end{align}
%
yielding
\begin{align}
p&= \frac{a^2+c^2-b^2}{2a}
\end{align}
%
From \eqref{eq:c_tricoord}, 
\begin{align}
\norm{\vec{A}}^2 &= c^2 = p^2+q^2
\\
\implies q&= \pm \sqrt{c^2-p^2}
\end{align}
%
The python code for  Fig. \ref{fig:tri_sss} is
\begin{lstlisting}
codes/triangle/tri_sss.py
\end{lstlisting}
%
and the equivalent latex-tikz code is
%
\begin{lstlisting}
figs/triangle/tri_sss.tex
\end{lstlisting}

%\end{enumerate}

