%\renewcommand{\theequation}{\theenumi}
%\begin{enumerate}[label=\thesubsection.\arabic*.,ref=\thesubsection.\theenumi]
%%\begin{enumerate}[label=\arabic*.,ref=\thesection.\theenumi]
%\numberwithin{equation}{enumi}


\item Draw a circle of radius 3 units. Take  two points $\vec{P}$ and $\vec{Q}$ on one of its extended 
diameter each at a distance of 7 units from its centre. Draw tangents to the circle from these two points 
$\vec{P}$ and $\vec{Q}$.
\item Draw a pair of tangents to a circle of radius 5 units which are inclined to each other at an angle of 
$60^{\degree}$.
\item Let ABC be a right triangle in which $a = 8, c = 6$ and $\angle B = 90^{\degree}$.  $BD$ is the 
perpendicular from $\vec{B}$ on $AC$ (altitude). The circle through $\vec{B}, \vec{C}, \vec{D}$ (circumcircle of $\triangle BCD$) is drawn.  Construct the 
tangents from $\vec{A}$ to this circle.

\item Draw a circle of diameter 6.1
\item Draw a circle with centre $\vec{B}$ and radius 6.  If $\vec{C}$ be  a point 10 units  away from its 
centre, construct the pair of tangents $AC$ and $CD$ to the 
circle.
\item Draw a circle of radius 3 and any two of its diameters.  Draw the ends of these diameters. What figure do you get?
\item Draw a line segment $AB$ of length 8 units. Taking $\vec{A}$ as centre, draw a circle of radius 4 units 
and taking $\vec{B}$ as centre, draw another circle of radius 3 units. Construct tangents to each circle from 
the centre of the other circle.

