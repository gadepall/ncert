%\renewcommand{\theequation}{\theenumi}
%\begin{enumerate}[label=\arabic*.,ref=\thesection.\theenumi]
%\numberwithin{equation}{enumi}
	%
%
% \renewcommand{\theequation}{\theenumi}
% \numberwithin{figure}{enumi}

\item Construct parallelogram $ABCD$ 	in Fig. \ref{fig:pgm_sas}	
given that  $BC = 5, AB = 6, \angle C = 85 \degree$.
\begin{figure}[!ht]
	\begin{center}
		\resizebox{\columnwidth}{!}{\input{./figs/quad/pgm_sas.tex}}
	\end{center}
	\caption{Parallelogram Properties}
	\label{fig:pgm_sas}	
\end{figure}
%
\\
\solution $BD$ is found using the cosine formula and $\triangle BDC$ is drawn using the approach in Construction \ref{const:tri_sss} with 
%
\begin{align}
\vec{B} = \myvec{0\\0},
\vec{C} = \myvec{5\\0},
\end{align}
%
Since the diagonals bisect each other, 
%
\begin{align}
\vec{O} &= \frac{\vec{B}+\vec{D}}{2}
\\
\vec{A} &= 2\vec{O} - \vec{C}.
\end{align}
%
$AB$ and $AD$ are then joined to complete the $\parallel$gm.
The python code for  Fig. \ref{fig:pgm_sas} is
\begin{lstlisting}
codes/quad/pgm_sas.py
\end{lstlisting}
%
and 
The equivalent latex-tikz code is
%
\begin{lstlisting}
figs/quad/pgm_sas.tex
\end{lstlisting}
%

\item Draw the $\parallel$gm $ABCD$ in 	Fig. \ref{fig:pgm_sss}	
with $BC = 6, CD = 4.5$ and $BD=7.5$.  Show that it is a rectangle.
\label{const:pgm_sss}
%
\begin{figure}[!ht]
	\begin{center}
		\resizebox{\columnwidth}{!}{\input{./figs/quad/pgm_sss.tex}}
	\end{center}
	\caption{Rectangle}
	\label{fig:pgm_sss}	
\end{figure}
\\
\solution It is easy to verify that 
%Using the approach in Construction\ref{const:tri_sss}, $\triangle BCD$ is drawn with
%
\begin{align}
BD^2=BC^2+C^2
\end{align}
%
Hence, using Baudhayana theorem, 
%
\begin{align}
\angle BCD = 90\degree
\end{align}
%
and  $ABCD$ is a rectangle.
\begin{align}
\vec{A} = \myvec{0\\4.5}
\vec{B} = \myvec{0\\0}
\vec{C} = \myvec{6\\0}
\vec{D} = \myvec{6\\4}
\end{align}
%
The python code for  Fig. \ref{fig:pgm_sss} is
\begin{lstlisting}
codes/quad/pgm_sss.py
\end{lstlisting}
%
and the equivalent latex-tikz code is
%
\begin{lstlisting}
figs/quad/pgm_sss.tex
\end{lstlisting}
%
%
%
%
%
\item Draw the rhombus $BEST$ with $BE = 4.5$ and $ET = 6$. 
\begin{figure}[!ht]
	\begin{center}
		\resizebox{\columnwidth}{!}{\input{./figs/quad/rhom_sss.tex}}
	\end{center}
	\caption{Rhombus}
	\label{fig:rhom_sss}	
\end{figure}
\\
\solution The coordinates of the various points in Fig. \ref{fig:rhom_sss} are obtained as
%
\begin{align}
\vec{O} = \myvec{0\\0},
\vec{B} = \myvec{0\\-4.5}
\\
\vec{E} = \myvec{3\\0},
\vec{S} = \myvec{4.5\\0},
\vec{T} = \myvec{0\\-3}
\end{align}
%
\item A square is a rectangle whose sides are equal.  Draw a square of side 4.5.
\\
\solution The coordinates of the various points in Fig. \ref{fig:square} are obtained as
%
\begin{align}
\vec{A} = \myvec{0\\4.5}
\\
\vec{B} = \myvec{0\\0},
\vec{C} = \myvec{4.5\\0},
\vec{D} = \myvec{4.5\\4.5}
\vec{O} = \frac{\vec{B}+\vec{C}}{2}
%
\end{align}
%
\begin{figure}[!ht]
	\begin{center}
		\resizebox{\columnwidth}{!}{\input{./figs/quad/square.tex}}
	\end{center}
	\caption{Square}
	\label{fig:square}	
\end{figure}

%
%\end{enumerate}
