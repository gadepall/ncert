\documentclass[journal,12pt,twocolumn]{IEEEtran}
%
\usepackage{setspace}
\usepackage{gensymb}
%\doublespacing
\singlespacing

%\usepackage{graphicx}
%\usepackage{amssymb}
%\usepackage{relsize}
\usepackage[cmex10]{amsmath}
%\usepackage{amsthm}
%\interdisplaylinepenalty=2500
%\savesymbol{iint}
%\usepackage{txfonts}
%\restoresymbol{TXF}{iint}
%\usepackage{wasysym}
\usepackage{amsthm}
\usepackage{iithtlc}
\usepackage{mathrsfs}
\usepackage{txfonts}
\usepackage{stfloats}
\usepackage{bm}
\usepackage{cite}
\usepackage{cases}
\usepackage{subfig}
%\usepackage{xtab}
\usepackage{longtable}
\usepackage{multirow}
%\usepackage{algorithm}
%\usepackage{algpseudocode}
\usepackage{booktabs}
\usepackage{enumitem}
\usepackage{mathtools}
\usepackage{tikz}
\usepackage{pgfplots}
\usepackage{circuitikz}
\usepackage{verbatim}
\usepackage{tfrupee}
\usepackage[breaklinks=true]{hyperref}
%\usepackage{stmaryrd}
\usepackage{tkz-euclide} % loads  TikZ and tkz-base
%\usetkzobj{all}
\usetikzlibrary{fit}
\usetikzlibrary{calc,math}
%\pgfdeclarelayer{background}
%\pgfsetlayers{background}
\usepackage{listings}
    \usepackage{color}                                            %%
    \usepackage{array}                                            %%
    \usepackage{longtable}                                        %%
    \usepackage{calc}                                             %%
    \usepackage{multirow}                                         %%
    \usepackage{hhline}                                           %%
    \usepackage{ifthen}                                           %%
  %optionally (for landscape tables embedded in another document): %%
    \usepackage{lscape}     
\usepackage{multicol}
\usepackage{chngcntr}
%\usepackage{enumerate}

%\usepackage{wasysym}
%\newcounter{MYtempeqncnt}
\DeclareMathOperator*{\Res}{Res}
%\renewcommand{\baselinestretch}{2}
\renewcommand\thesection{\arabic{section}}
\renewcommand\thesubsection{\thesection.\arabic{subsection}}
\renewcommand\thesubsubsection{\thesubsection.\arabic{subsubsection}}

\renewcommand\thesectiondis{\arabic{section}}
\renewcommand\thesubsectiondis{\thesectiondis.\arabic{subsection}}
\renewcommand\thesubsubsectiondis{\thesubsectiondis.\arabic{subsubsection}}

% correct bad hyphenation here
\hyphenation{op-tical net-works semi-conduc-tor}
\def\inputGnumericTable{}                                 %%

\lstset{
%language=C,
frame=single, 
breaklines=true,
columns=fullflexible
}
%\lstset{
%language=tex,
%frame=single, 
%breaklines=true
%}

\begin{document}
%


\newtheorem{theorem}{Theorem}[section]
\newtheorem{problem}{Problem}
\newtheorem{proposition}{Proposition}[section]
\newtheorem{lemma}{Lemma}[section]
\newtheorem{corollary}[theorem]{Corollary}
\newtheorem{example}{Example}[section]
\newtheorem{definition}[problem]{Definition}
%\newtheorem{thm}{Theorem}[section] 
%\newtheorem{defn}[thm]{Definition}
%\newtheorem{algorithm}{Algorithm}[section]
%\newtheorem{cor}{Corollary}
\newcommand{\BEQA}{\begin{eqnarray}}
\newcommand{\EEQA}{\end{eqnarray}}
\newcommand{\define}{\stackrel{\triangle}{=}}
\newcommand{\RomanNumeralCaps}[1]
    {\MakeUppercase{\romannumeral #1}}
\bibliographystyle{IEEEtran}
%\bibliographystyle{ieeetr}


\providecommand{\mbf}{\mathbf}
\providecommand{\pr}[1]{\ensuremath{\Pr\left(#1\right)}}
\providecommand{\qfunc}[1]{\ensuremath{Q\left(#1\right)}}
\providecommand{\sbrak}[1]{\ensuremath{{}\left[#1\right]}}
\providecommand{\lsbrak}[1]{\ensuremath{{}\left[#1\right.}}
\providecommand{\rsbrak}[1]{\ensuremath{{}\left.#1\right]}}
\providecommand{\brak}[1]{\ensuremath{\left(#1\right)}}
\providecommand{\lbrak}[1]{\ensuremath{\left(#1\right.}}
\providecommand{\rbrak}[1]{\ensuremath{\left.#1\right)}}
\providecommand{\cbrak}[1]{\ensuremath{\left\{#1\right\}}}
\providecommand{\lcbrak}[1]{\ensuremath{\left\{#1\right.}}
\providecommand{\rcbrak}[1]{\ensuremath{\left.#1\right\}}}
\theoremstyle{remark}
\newtheorem{rem}{Remark}
\newcommand{\sgn}{\mathop{\mathrm{sgn}}}
\providecommand{\abs}[1]{\left\vert#1\right\vert}
\providecommand{\res}[1]{\Res\displaylimits_{#1}} 
\providecommand{\norm}[1]{\left\lVert#1\right\rVert}
%\providecommand{\norm}[1]{\lVert#1\rVert}
\providecommand{\mtx}[1]{\mathbf{#1}}
\providecommand{\mean}[1]{E\left[ #1 \right]}
\providecommand{\fourier}{\overset{\mathcal{F}}{ \rightleftharpoons}}
%\providecommand{\hilbert}{\overset{\mathcal{H}}{ \rightleftharpoons}}
\providecommand{\system}{\overset{\mathcal{H}}{ \longleftrightarrow}}
	%\newcommand{\solution}[2]{\textbf{Solution:}{#1}}
\newcommand{\solution}{\noindent \textbf{Solution: }}
\newcommand{\cosec}{\,\text{cosec}\,}
\providecommand{\dec}[2]{\ensuremath{\overset{#1}{\underset{#2}{\gtrless}}}}
\newcommand{\myvec}[1]{\ensuremath{\begin{pmatrix}#1\end{pmatrix}}}
\newcommand{\mydet}[1]{\ensuremath{\begin{vmatrix}#1\end{vmatrix}}}
\newcommand*{\permcomb}[4][0mu]{{{}^{#3}\mkern#1#2_{#4}}}
\newcommand*{\perm}[1][-3mu]{\permcomb[#1]{P}}
\newcommand*{\comb}[1][-1mu]{\permcomb[#1]{C}}

%\newcommand*{\perm}[2]{{}^{#1}\!P_{#2}}%
%\newcommand*{\comb}[2]{{}^{#1}C_{#2}}%
%\numberwithin{equation}{section}
\numberwithin{equation}{subsection}
%\numberwithin{problem}{section}
%\numberwithin{definition}{section}
\makeatletter
\@addtoreset{figure}{problem}
\makeatother

\let\StandardTheFigure\thefigure
\let\vec\mathbf
%\renewcommand{\thefigure}{\theproblem.\arabic{figure}}
\renewcommand{\thefigure}{\theproblem}
%\setlist[enumerate,1]{before=\renewcommand\theequation{\theenumi.\arabic{equation}}
%\counterwithin{equation}{enumi}


%\renewcommand{\theequation}{\arabic{subsection}.\arabic{equation}}

\def\putbox#1#2#3{\makebox[0in][l]{\makebox[#1][l]{}\raisebox{\baselineskip}[0in][0in]{\raisebox{#2}[0in][0in]{#3}}}}
     \def\rightbox#1{\makebox[0in][r]{#1}}
     \def\centbox#1{\makebox[0in]{#1}}
     \def\topbox#1{\raisebox{-\baselineskip}[0in][0in]{#1}}
     \def\midbox#1{\raisebox{-0.5\baselineskip}[0in][0in]{#1}}

\vspace{3cm}

\title{
%	\logo{
Probability 
%	}
}
\author{ G V V Sharma$^{*}$% <-this % stops a space
	\thanks{*The author is with the Department
		of Electrical Engineering, Indian Institute of Technology, Hyderabad
		502285 India e-mail:  gadepall@iith.ac.in. All content in this manual is released under GNU GPL.  Free and open source.}
	
}	
%\title{
%	\logo{Matrix Analysis through Octave}{\begin{center}\includegraphics[scale=.24]{tlc}\end{center}}{}{HAMDSP}
%}


% paper title
% can use linebreaks \\ within to get better formatting as desired
%\title{Matrix Analysis through Octave}
%
%
% author names and IEEE memberships
% note positions of commas and nonbreaking spaces ( ~ ) LaTeX will not break
% a structure at a ~ so this keeps an author's name from being broken across
% two lines.
% use \thanks{} to gain access to the first footnote area
% a separate \thanks must be used for each paragraph as LaTeX2e's \thanks
% was not built to handle multiple paragraphs
%

%\author{<-this % stops a space
%\thanks{}}
%}
% note the % following the last \IEEEmembership and also \thanks - 
% these prevent an unwanted space from occurring between the last author name
% and the end of the author line. i.e., if you had this:
% 
% \author{....lastname \thanks{...} \thanks{...} }
%                     ^------------^------------^----Do not want these spaces!
%
% a space would be appended to the last name and could cause every name on that
% line to be shifted left slightly. This is one of those "LaTeX things". For
% instance, "\textbf{A} \textbf{B}" will typeset as "A B" not "AB". To get
% "AB" then you have to do: "\textbf{A}\textbf{B}"
% \thanks is no different in this regard, so shield the last } of each \thanks
% that ends a line with a % and do not let a space in before the next \thanks.
% Spaces after \IEEEmembership other than the last one are OK (and needed) as
% you are supposed to have spaces between the names. For what it is worth,
% this is a minor point as most people would not even notice if the said evil
% space somehow managed to creep in.



% The paper headers
%\markboth{Journal of \LaTeX\ Class Files,~Vol.~6, No.~1, January~2007}%
%{Shell \MakeLowercase{\textit{et al.}}: Bare Demo of IEEEtran.cls for Journals}
% The only time the second header will appear is for the odd numbered pages
% after the title page when using the twoside option.
% 
% *** Note that you probably will NOT want to include the author's ***
% *** name in the headers of peer review papers.                   ***
% You can use \ifCLASSOPTIONpeerreview for conditional compilation here if
% you desire.




% If you want to put a publisher's ID mark on the page you can do it like
% this:
%\IEEEpubid{0000--0000/00\$00.00~\copyright~2007 IEEE}
% Remember, if you use this you must call \IEEEpubidadjcol in the second
% column for its text to clear the IEEEpubid mark.



% make the title area
\maketitle

\newpage

\tableofcontents

\bigskip

\renewcommand{\thefigure}{\theenumi}
\renewcommand{\thetable}{\theenumi}
%\renewcommand{\theequation}{\theenumi}

%\begin{abstract}
%%\boldmath
%In this letter, an algorithm for evaluating the exact analytical bit error rate  (BER)  for the piecewise linear (PL) combiner for  multiple relays is presented. Previous results were available only for upto three relays. The algorithm is unique in the sense that  the actual mathematical expressions, that are prohibitively large, need not be explicitly obtained. The diversity gain due to multiple relays is shown through plots of the analytical BER, well supported by simulations. 
%
%\end{abstract}
% IEEEtran.cls defaults to using nonbold math in the Abstract.
% This preserves the distinction between vectors and scalars. However,
% if the journal you are submitting to favors bold math in the abstract,
% then you can use LaTeX's standard command \boldmath at the very start
% of the abstract to achieve this. Many IEEE journals frown on math
% in the abstract anyway.

% Note that keywords are not normally used for peerreview papers.
%\begin{IEEEkeywords}
%Cooperative diversity, decode and forward, piecewise linear
%\end{IEEEkeywords}



% For peer review papers, you can put extra information on the cover
% page as needed:
% \ifCLASSOPTIONpeerreview
% \begin{center} \bfseries EDICS Category: 3-BBND \end{center}
% \fi
%
% For peerreview papers, this IEEEtran command inserts a page break and
% creates the second title. It will be ignored for other modes.
%\IEEEpeerreviewmaketitle

\begin{abstract}
This book provides a computational approach to probability and statistics based on the NCERT textbooks from Class 6-12.  Links to sample Python codes are available in the text.  
\end{abstract}
Download python codes using 
\begin{lstlisting}
svn co https://github.com/gadepall/school/trunk/ncert/probability/codes
\end{lstlisting}

\section{Bernoulli Distribution}
\renewcommand{\theequation}{\theenumi}
\begin{enumerate}[label=\thesection.\arabic*.,ref=\thesection.\theenumi]
\numberwithin{equation}{enumi}


\item A jar contains 24 marbles, some are green and others are blue. If a marble is drawn at random from the jar, the probability that it is green is $\frac{2}{3}$. Find the number of blue balls in the jar.
\\
\solution
 Let the random variable $X = \{ 0,1 \}$ denote the outcome of the given experiment.
\\$X = 1$ if the marble picked turns out $Green$.
\\$X = 0$ if the marble picked turns out $Blue$.
\\It is given that,
\begin{align}
P(X = 1) &= \frac{2}{3}\\
\implies P(X = 0) &= 1 - P(X = 1)\\
\implies P(X = 0) &= 1 - \frac{2}{3}\\
\implies P(X = 0) &= \frac{1}{3}
\end{align}
Now
\begin{align}
n(X = 0) + n(X = 1) &= 24\\
\because P(X = 0) &= \frac{n(X = 0)}{n(X = 0) + n(X = 1)},\\
 n(X = 0) &= P(X = 0)\left(n(X = 0) + n(X = 1)\right)\\
\implies n(X = 0) &= \frac{(1)\times (24)}{3}\\
\implies n(X = 0) &= 8
\end{align}
$\therefore{}$ the  number of blue balls is 8.

\item A bag contains lemon flavoured candies only. Malini takes out one candy without
looking into the bag. What is the probability that she takes out\\
(i) an orange flavoured candy?\\
(ii) a lemon flavoured candy?
\solution  Let the random variable $X=\{0,1\}$ represent the outcome of the flavour of the candy Malini picks. $X=0$ denotes an orange flavoured candy, while $X=1$ denotes a lemon flavoured candy.  Then 
\begin{align}
\pr{X=0} = 1,
\\
\pr{X=1} = 0
\end{align}
\item 
\item 
\item 
\item 
\item 
\item A person buys a lottery ticket in 50 lotteries, in each of which his chance of
winning a prize is $\frac{1}{100}$.What is the probability that he will win a prize\\
(a) at least once \\
(b) exactly once \\
(c) at least twice?\\
From the given information, the random variable representing the trials is
\begin{align}
X \sim B\brak{50,\frac{1}{100}}
\end{align}
%
Hence the desired probabilites are
\begin{enumerate}
\item 
\begin{align}
\pr{X \ge 1} = 1 - \pr{X =0} = 1-\brak{\frac{99}{100}}^{50} 
\end{align}
%
\item 
\begin{align}
\pr{X = 1} = {50}\brak{\frac{99}{100}}^{49} \brak{\frac{1}{100}}
\end{align}
%
\item 
\begin{align}
\pr{X \ge 2} &= 1 - \pr{X \le 1} 
\\
&= 1 - \pr{X =0} - \pr{X = 1} \\
&= 1 -\brak{\frac{149}{100}}\brak{\frac{99}{100}}^{49}\\
&= 0.0894
\end{align}

\end{enumerate}


\item In an examination, 20 questions of true-false type are asked. Suppose a student tosses a fair coin to determine his answer to each question. If the coin falls heads, he answers 'true'; if it falls tails, he answers 'false'. Find the probability that he answers at least 12 questions correctly.\\
\solution
Let X be the number of correct answers\\
n be the number of questions $(n=20)$\\
p is the probability of correct answer $(p=0.5)$\\
q is the probability of wrong answer $(q=1-p)$
From Bernoulli's distribution,
\begin{align}
    \pr{X=r}&=\comb{n}{r}p^rq^{n-r}
\end{align}
$\therefore$ required probability is
\begin{align}
    \pr{X\geq12}&=\sum_{r=12}^{n}\comb{n}{r} p^r q^{n-r}\\
    &=\sum_{r=12}^{20}\comb{20}{r} p^r (1-p)^{20-r}\\
    &=0.25172233581
\end{align}

\item There are 5$\%$ defective items in a large bulk of items. What is the probability
that a sample of 10 items will include not more than one defective item?\\
\solution
Let X be the random variable representing all the items.  Then,
\begin{align}
X = \sum_{i=1}^{10}X_i
\end{align}
%
has a Binomial distribution with
$X_i \in \cbrak{0,1}$ being a Bernoulli r.v. representing the item condition.  From the given information, the probability of an item being
defective is given by
\begin{align}
\pr{X_i = 0} &= \frac{1}{20} = p
\\
\implies \pr{X_i = 1} &= q = 1-p =  1-\frac{1}{20} = \frac{19}{20}
\end{align}
%
$\because$
 \begin{align}
    X &\sim B(n=10,p=0.5),
    \\
   \pr{X=r}&=    \comb{n}{r} p^r q^{n-r}
   \end{align}
\begin{multline}
\implies Pr(X \le 1)=Pr(X=0)+Pr(X=1) \\
              =\comb{10}{0}\left(\frac{1}{20}\right)^0 \left(\frac{19}{20}\right)^{10}+\comb{10}{1} \left(\frac{1}{20}\right)^1 \left(\frac{19}{20}\right)^{9}\\
              =\left(\frac{29}{20}\right) \times\left (\frac{19}{20}\right) ^{9}
              =0.9138
\end{multline} 



\item In a meeting, 70$\%$ of the members favour and 30$\%$ oppose a certain proposal.
A member is selected at random and we take X = 0 if he opposed, and X = 1 if he is in favour. Find E(X) and Var (X).\\
\solution
From the given information,
\begin{align}
\pr{X=0} = 70 {\%} = 0.7\\
\pr{X=1} = 30 {\%} =0.3
\end{align}
Hence,
\begin{align}
    E(X) &= 1\times0.7 + 0\times0.3 =0.7
    \\
    E(X^2) &= 1^2\times0.7 + 0^2\times0.3 =0.7
    \\
    \implies Var(X) &= E(X^2) -[E(X)]^2
    \\&= 0.7- 0.7^2 = 0.21
\end{align}

\item A coin is biased so that the head is 3 times as likely to occur as tail. If the coin is tossed twice, find the probability distribution of number of tails.\\
\item From a lot of 30 bulbs which include 6 defectives, a sample of 4 bulbs is drawn at random with replacement. Find the probability distribution of the number of defective bulbs.\\
\item Probability that A speaks truth is $\frac{4}{5}$. A coin is tossed. A reports that a head appears. The probability that actually there was head is\\
\begin{enumerate}
\item $\frac{4}{5}$
\item $\frac{1}{2}$
\item $\frac{1}{5}$
\item $\frac{2}{5}$
\end{enumerate}
\solution
Let X $\in\{0,1\}$ be the random variable denoting that A tells truth when X=1

\begin{align*}
\tag{1.14.1}
     \Pr(X=1)=\frac{4}{5}\\
      \Pr(X=0)=1-\Pr(X=1)
\end{align*}
\begin{align}
\tag{1.14.2}
    \Pr(X=0) =\frac{1}{5}
\end{align}
Let Y $\in\{0,1\}$ be the random variable denoting that Head appears on the coin when Y=1 \\
As the coin is unbiased,\\
\begin{align}
\tag{1.14.3}
    \Pr(Y=1|X=1)=\frac{1}{2}
\end{align}
\begin{align}
\tag{1.14.4}
    \Pr(Y=1|X=0)=\frac{1}{2}
\end{align}
Probability that actually there was a head given that A reports a Head\\
=$\Pr(X=1|Y=1)$

From Bayes Theorm,

\begin{align*}
    &\Pr(X=1|Y=1)=\frac{\Pr(X=1)\times\Pr(Y=1|X=1)}{\sum_{i=0}^{1}\Pr(X=i)\times\Pr(Y=1|X=i)}\\
   &=\frac{\frac{4}{5}\times\frac{1}{2}}{\frac{4}{5}\times\frac{1}{2}+\frac{1}{5}\times\frac{1}{2}}\\
    &=\frac{4}{5}
\end{align*}
Probability that  actually there was a head given that A reports a Head=$\frac{4}{5}$\\
So, option a) is correct.

\item A box of oranges is inspected by examining three randomly selected oranges drawn without replacement. If all the three oranges are good, the box is approved for sale, otherwise, it is rejected. Find the probability that a box containing 15 oranges out of which 12 are good and 3 are bad ones will be approved for sale.\\
\solution
%
Let the $i$th inspection be $X_i \in \cbrak{0,1}$, where  1 represents a good orange.  From the given information,
\begin{align}
\pr{X_1=1}&=\brak{\frac{12}{15}}\\ 
\pr{X_2=1|X_1=1}&=\brak{\frac{11}{14}} \\ 
\pr{X_3=1|X_1=1,X_2=1} &= \brak{\frac{10}{13}}
\end{align}
%
The probability that the box will be approved for sale is
\begin{multline}
\pr{X_1=1,X_2=1,X_3=1} 
\\
= \pr{X_1=1} \times \pr{X_2=1|X_1=1} 
\\
\times \pr{X_3=1|X_1=1,X_2=1} \\ 
=\frac{12}{15}\times \frac{11}{14} \times \frac{10}{13}\\ 
=\frac{1320}{2730} =0.483 
\end{multline}
%
\item Determine P(E/F), if a coin is tossed three times\\
(i) E : head on third toss , F : heads on first two tosses\\
(ii) E : at least two heads , F : at most two heads\\
(iii) E : at most two tails , F : at least one tail\\
%
\solution
In an experiment of tossing a coin $n$( = 3) times, random variable  $X \in \lbrace 0,1,2,3 \rbrace$ follows binomial distribution.\\
The binomial distribution formula is:
\begin{align*}
 \Pr( X=k ) &= \comb{n}{k} \times p^k \times (1- p)^{n - k}
\end{align*}

Where:


\begin{table}[h]

    \centering
    \resizebox{\columnwidth}{!}{%
    \begin{tabular}{|r|c|}\hline
    $k$ &  total number of “successes” \\ \hline
    $p$ & probability of a success on an individual trial\\ \hline
    $n$ & number of trials = 3 \\ \hline
\end{tabular}}
\caption{The binomial distribution formula}
    \label{1.16:table:0}
\end{table}


\begin{enumerate}[label=(\roman*)]
    \item From table \ref{1.16:table:1}, $\Pr(E|F)$ = 0.5
    \item $X$ denotes number of heads. From table \ref{1.16:table:2}, $\Pr(E|F)$ = 0.428
    \item $X$ denotes number of tails. From table \ref{1.16:table:3}, $\Pr(E|F)$ = 0.857
\end{enumerate}


\begin{table}[ht]

    \centering
     \resizebox{\columnwidth}{!}{%
    \begin{tabular}{|r|l|} \hline
    $\Pr$(Event)  & Calculation   \\ \hline
    $\Pr( F)$    & From product rule , \\ 
    &= $\frac{1}{2}\times\frac{1}{2}$ \\ 
    &=  0.25 \\ \hline 
    $\Pr( EF)$   &  From product rule, \\   
    &= $\frac{1}{2}\times\frac{1}{2}\times\frac{1}{2}$ \\ 
    & = 0.125\\ \hline 
    $\Pr(E|F )$  &= $\frac{\Pr(EF)}{\Pr(F)} $ \\ 
    &  = 0.5 \\ \hline 
    \end{tabular} }
    \caption{Part(i)}
    \label{1.16:table:1}
\end{table}

\begin{table}[ht]

    \centering
    \resizebox{\columnwidth}{!}{%
     \begin{tabular}{|r|l|}\hline
      $\Pr$(Event)  & Calculation \\ \hline
      $\Pr( F)$ 
      &= $\Pr( X\leq2)$ \\ 
      &= $ \Pr( X=0) + \Pr( X=1) + \Pr( X=2 )$\\ 
      &= $\comb{3}{0} \left(\frac{1}{2}\right)^3  + \comb{3}{1} \left(\frac{1}{2}\right)^3 + \comb{3}{2} \left(\frac{1}{2}\right)^3$\\ 
      &= 0.875 \\ \hline
    $\Pr( EF)$  &= $\Pr( X=2) $ \\
    &= 0.375 \\ \hline
    $\Pr( E|F )$   &=$ \frac{\Pr(EF)}{\Pr(F)} $ \\ 
    &= 0.428 \\ \hline
    \end{tabular}}
    \caption{Part(ii)}
    \label{1.16:table:2}
\end{table}


\begin{table}[ht]

       \centering
       \resizebox{\columnwidth}{!}{%
       \begin{tabular}{|r|l|}\hline
      $\Pr$(Event)  & Calculation \\ \hline
       $\Pr( F)$ &= $\Pr( X\geq1)$\\
        &= $1-\Pr( 0)$ \\
        &= 0.875 \\ \hline
        $\Pr( EF)$ &= $\Pr(X= 1) + \Pr(X= 2 )$ \\
         &= 0.75 \\ \hline
         $\Pr( E|F)$ &= $\frac{\Pr(EF)}{\Pr(F)} $ \\
        &= 0.857 \\ \hline
      \end{tabular} }
       \caption{Part(iii)}
       \label{1.16:table:3}
   \end{table}
  


\item Determine P(E/F), if two coins are tossed once, where\\
(i) E : tail appears on one coin, F : one coin shows head\\
(ii) E : no tail appears, F : no head appears\\
%
\solution
Let X denote the number of heads shown on the coins, where n = 2 and p = 0.5, q = 1-p
\begin{align}
p(x) = \Pr(X=x) = \binom{n}{x}\times p^x\times q^{n-x}
\end{align}
\begin{table}[!ht]
\resizebox{\columnwidth}{!} {
\begin{tabular}{|c|c|c|c|}
\hline
     X&0&1&2  \\
     \hline
     P(X)&$\binom{2}{0}(0.5)^2=\dfrac{1}{4}$&$\binom{2}{1}(0.5)^2 = \dfrac{1}{2}$&$\binom{2}{2}(0.5)^2 = \dfrac{1}{4}$\\
     \hline
\end{tabular}
}
\caption{Probability of number of heads shown on the coins }
\label{1.17:table:1}
\end{table}
\begin{enumerate}
\item 
\begin{align}
&\Pr(F) = \Pr(X\geq 1)\\
&\Pr(F) = \Pr(X = 1) + \Pr(X = 2)\nonumber\\
       &= \frac{1}{2} + \frac{1}{4} = \frac{3}{4}\\
&\Pr(EF) = \Pr(X = 1) = \frac{1}{2}\\
&\Pr(E/F) =\frac{\Pr(EF)}{\Pr(F)} = \frac{2}{3}
\end{align}
\item
\begin{align}
&\Pr(F) = \Pr(X = 0) = \frac{1}{4}\\
&\Pr(EF) = 0\\
&\Pr(E/F) = \frac{\Pr(EF)}{\Pr(F)} = 0
\end{align} 
\end{enumerate}

\item Two players, Sangeeta and Reshma, play a tennis match. It is known
that the probability of Sangeeta winning the match is 0.62. What is the probability of
Reshma winning the match?
%
\\
\solution The desired probability is $1 - 0.62 = 0.38$.
\item Harpreet tosses two different coins simultaneously (say, one is of rupee 1
and other of rupee 2). What is the probability that she gets at least one head?

	\item In a cricket match, a batswoman hits a boundary 6 times out of 30 balls she plays. Find the probability that she did not hit a boundary.
\\
\solution
%
\begin{align}
    E(X) &= \frac{1}{\sqrt{2\pi}} \int_{-\infty}^{\infty} x e^{-\frac{x^2}{2}}dx\\
    &=0 \quad \brak{ \text{ odd function}}
\end{align}
\begin{align}
    E\brak{X^2}&= \frac{1}{\sqrt{2\pi}}\int_{-\infty}^{\infty} x^2
e^ {-\frac{x^2}{2}} dx \quad \brak{even function}\\
    &= \frac{2}{\sqrt{2\pi}} \int_{0}^{\infty} x^2 e^{-\frac{x^2}{2}} dx\\
    &= \frac{2}{\sqrt{2\pi}}\int_{0}^{\infty}\sqrt{2u}e^{-u} du \quad\brak{Let \frac{x^2}{2}= u}\\
    &= \frac{2}{\sqrt{\pi}} \int_{0}^{\infty} e^{-u} u^{\frac{3}{2}-1} du\\
    &= \frac{2}{\sqrt{\pi}} \Gamma\brak{{\frac{3}{2}}}\\
    &= \frac{1}{\sqrt{\pi}}\Gamma\brak{\frac{1}{2}} \\
    &= 1
\end{align}
where we have used the fact that
\begin{align}
\quad\because \Gamma(n)= (n-1)\Gamma(n-1); \Gamma\brak{\frac{1}{2}}=\sqrt{\pi}
\end{align}
%
Thus, the  variance is
\begin{align}
    \sigma^2 =  E\brak X^2 - E^2\brak X = 1
\end{align}

	\item A coin is tossed 1000 times with the following frequencies:\\
Head : 455, Tail : 545\\
Compute the probability for each event.\\
\solution
\input{./solutions/1-10/chapters/prob_exm/prob1/solution1.tex}
   \item Two coins are tossed simultaneously 500 times, and we get\\
       Two heads : 105 times\\
       One head : 275 times\\
       No head : 120 times\\
Find the probability of occurrence of each of these events.\\
\solution
\input{./solutions/1-10/chapters/prob_exm/prob2/solution2.tex}
   \item A die is thrown 1000 times with the frequencies for the outcomes 1, 2, 3, 4, 5 and 6 as given in the following Table \ref{table:prob_exam_3}.
Find the probability of getting each outcome.

\begin{table}[!ht]
\centering
\resizebox{\columnwidth}{!}{
\begin{tabular}{ |c|c|c|c|c|c|c| } 
 \hline
 \textbf{Outcome} &1 &2 &3 &4 &5 &6  \\ 
 \hline
 \textbf{Frequency} &179 &150 &157 &149 &175 &190 \\ 
 \hline
\end{tabular}
}
\caption{}
\label{table:prob_exam_3}
\end{table}
%\\
\solution
\input{./solutions/1-10/chapters/prob_exm/prob3/solution3.tex}

   \item The record of a weather station shows that out of the past 250 consecutive days, its weather forecasts were correct 175 times.\\
   (i) What is the probability that on a given day it was correct?\\
(ii) What is the probability that it was not correct on a given day?\\
\solution
\input{./solutions/1-10/chapters/prob_exm/prob5/solution5.tex}


\item {\em: Random Process}A and B throw a die alternatively till one of them gets a '6' and wins the game. Find their respective probabilities of winning, if A starts first.
\\
\solution
%
\begin{align}
    E(X) &= \frac{1}{\sqrt{2\pi}} \int_{-\infty}^{\infty} x e^{-\frac{x^2}{2}}dx\\
    &=0 \quad \brak{ \text{ odd function}}
\end{align}
\begin{align}
    E\brak{X^2}&= \frac{1}{\sqrt{2\pi}}\int_{-\infty}^{\infty} x^2
e^ {-\frac{x^2}{2}} dx \quad \brak{even function}\\
    &= \frac{2}{\sqrt{2\pi}} \int_{0}^{\infty} x^2 e^{-\frac{x^2}{2}} dx\\
    &= \frac{2}{\sqrt{2\pi}}\int_{0}^{\infty}\sqrt{2u}e^{-u} du \quad\brak{Let \frac{x^2}{2}= u}\\
    &= \frac{2}{\sqrt{\pi}} \int_{0}^{\infty} e^{-u} u^{\frac{3}{2}-1} du\\
    &= \frac{2}{\sqrt{\pi}} \Gamma\brak{{\frac{3}{2}}}\\
    &= \frac{1}{\sqrt{\pi}}\Gamma\brak{\frac{1}{2}} \\
    &= 1
\end{align}
where we have used the fact that
\begin{align}
\quad\because \Gamma(n)= (n-1)\Gamma(n-1); \Gamma\brak{\frac{1}{2}}=\sqrt{\pi}
\end{align}
%
Thus, the  variance is
\begin{align}
    \sigma^2 =  E\brak X^2 - E^2\brak X = 1
\end{align}

%
\item To know the opinion of the students about the subject statistics, a survey of 200 students was conducted. The data is recorded in Table \ref{table:1.2.6}
Find the probability that a student chosen at random
\begin{enumerate}
\item likes statistics,
\item  does not like it.
\end{enumerate}
\begin{table}[!ht]
\centering
\begin{tabular}{ |c|c| } 
 \hline
 \textbf{Opinion} &\textbf{Number of students}\\
 \hline
 like  &135\\ 
 \hline
 dislike  &65\\ 
 \hline
\end{tabular}
\caption{}
\label{table:1.2.6}
\end{table}
\solution
%
\begin{align}
    E(X) &= \frac{1}{\sqrt{2\pi}} \int_{-\infty}^{\infty} x e^{-\frac{x^2}{2}}dx\\
    &=0 \quad \brak{ \text{ odd function}}
\end{align}
\begin{align}
    E\brak{X^2}&= \frac{1}{\sqrt{2\pi}}\int_{-\infty}^{\infty} x^2
e^ {-\frac{x^2}{2}} dx \quad \brak{even function}\\
    &= \frac{2}{\sqrt{2\pi}} \int_{0}^{\infty} x^2 e^{-\frac{x^2}{2}} dx\\
    &= \frac{2}{\sqrt{2\pi}}\int_{0}^{\infty}\sqrt{2u}e^{-u} du \quad\brak{Let \frac{x^2}{2}= u}\\
    &= \frac{2}{\sqrt{\pi}} \int_{0}^{\infty} e^{-u} u^{\frac{3}{2}-1} du\\
    &= \frac{2}{\sqrt{\pi}} \Gamma\brak{{\frac{3}{2}}}\\
    &= \frac{1}{\sqrt{\pi}}\Gamma\brak{\frac{1}{2}} \\
    &= 1
\end{align}
where we have used the fact that
\begin{align}
\quad\because \Gamma(n)= (n-1)\Gamma(n-1); \Gamma\brak{\frac{1}{2}}=\sqrt{\pi}
\end{align}
%
Thus, the  variance is
\begin{align}
    \sigma^2 =  E\brak X^2 - E^2\brak X = 1
\end{align}

\item  Assume that each born child is equally likely to be a boy or a girl. If a family has two children, what is the conditional probability that both are girls given that\\
(i) the youngest is a girl,\\ 
(ii) at least one is a girl?\\
\solution
\input{./solutions/20-30/chapters/prob/exercises/docq23.tex}

\item  An instructor has a question bank consisting of 300 easy True / False questions,
200 difficult True / False questions, 500 easy multiple choice questions and 400 difficult multiple choice questions. If a question is selected at random from the question bank, what is the probability that it will be an easy question given that it is a multiple choice question?\\
\solution
\input{./solutions/20-30/chapters/prob/exercises/docq24.tex}
\item Two cards are drawn at random and without replacement from a pack of 52 playing cards. Find the probability that both the cards are black.\\
\solution
\input{./solutions/20-30/chapters/prob/exercises/docq30.tex}
\item Two balls are drawn at random with replacement from a box containing 10 black and 8 red balls. Find the probability that\\
(i) both balls are red.\\
(ii) first ball is black and second is red.\\
(iii) one of them is black and other is red.\\
\solution
%
\begin{align}
    E(X) &= \frac{1}{\sqrt{2\pi}} \int_{-\infty}^{\infty} x e^{-\frac{x^2}{2}}dx\\
    &=0 \quad \brak{ \text{ odd function}}
\end{align}
\begin{align}
    E\brak{X^2}&= \frac{1}{\sqrt{2\pi}}\int_{-\infty}^{\infty} x^2
e^ {-\frac{x^2}{2}} dx \quad \brak{even function}\\
    &= \frac{2}{\sqrt{2\pi}} \int_{0}^{\infty} x^2 e^{-\frac{x^2}{2}} dx\\
    &= \frac{2}{\sqrt{2\pi}}\int_{0}^{\infty}\sqrt{2u}e^{-u} du \quad\brak{Let \frac{x^2}{2}= u}\\
    &= \frac{2}{\sqrt{\pi}} \int_{0}^{\infty} e^{-u} u^{\frac{3}{2}-1} du\\
    &= \frac{2}{\sqrt{\pi}} \Gamma\brak{{\frac{3}{2}}}\\
    &= \frac{1}{\sqrt{\pi}}\Gamma\brak{\frac{1}{2}} \\
    &= 1
\end{align}
where we have used the fact that
\begin{align}
\quad\because \Gamma(n)= (n-1)\Gamma(n-1); \Gamma\brak{\frac{1}{2}}=\sqrt{\pi}
\end{align}
%
Thus, the  variance is
\begin{align}
    \sigma^2 =  E\brak X^2 - E^2\brak X = 1
\end{align}


\item Probability of solving specific problem independently by A and B are $\frac{1}{2}$
and $\frac{1}{3}$ respectively. If both try to solve the problem independently, find the probability that\\
(i) the problem is solved \\
(ii) exactly one of them solves the problem.\\
\solution
%
\begin{align}
    E(X) &= \frac{1}{\sqrt{2\pi}} \int_{-\infty}^{\infty} x e^{-\frac{x^2}{2}}dx\\
    &=0 \quad \brak{ \text{ odd function}}
\end{align}
\begin{align}
    E\brak{X^2}&= \frac{1}{\sqrt{2\pi}}\int_{-\infty}^{\infty} x^2
e^ {-\frac{x^2}{2}} dx \quad \brak{even function}\\
    &= \frac{2}{\sqrt{2\pi}} \int_{0}^{\infty} x^2 e^{-\frac{x^2}{2}} dx\\
    &= \frac{2}{\sqrt{2\pi}}\int_{0}^{\infty}\sqrt{2u}e^{-u} du \quad\brak{Let \frac{x^2}{2}= u}\\
    &= \frac{2}{\sqrt{\pi}} \int_{0}^{\infty} e^{-u} u^{\frac{3}{2}-1} du\\
    &= \frac{2}{\sqrt{\pi}} \Gamma\brak{{\frac{3}{2}}}\\
    &= \frac{1}{\sqrt{\pi}}\Gamma\brak{\frac{1}{2}} \\
    &= 1
\end{align}
where we have used the fact that
\begin{align}
\quad\because \Gamma(n)= (n-1)\Gamma(n-1); \Gamma\brak{\frac{1}{2}}=\sqrt{\pi}
\end{align}
%
Thus, the  variance is
\begin{align}
    \sigma^2 =  E\brak X^2 - E^2\brak X = 1
\end{align}

\item (i) A lot of 20 bulbs contain 4 defective ones. One bulb is drawn at random from the lot.
What is the probability that this bulb is defective?\\
(ii) Suppose the bulb drawn in (i) is not defective and is not replaced. Now one bulb
is drawn at random from the rest. What is the probability that this bulb is not
defective ?
\\
\solution
%
\begin{align}
    E(X) &= \frac{1}{\sqrt{2\pi}} \int_{-\infty}^{\infty} x e^{-\frac{x^2}{2}}dx\\
    &=0 \quad \brak{ \text{ odd function}}
\end{align}
\begin{align}
    E\brak{X^2}&= \frac{1}{\sqrt{2\pi}}\int_{-\infty}^{\infty} x^2
e^ {-\frac{x^2}{2}} dx \quad \brak{even function}\\
    &= \frac{2}{\sqrt{2\pi}} \int_{0}^{\infty} x^2 e^{-\frac{x^2}{2}} dx\\
    &= \frac{2}{\sqrt{2\pi}}\int_{0}^{\infty}\sqrt{2u}e^{-u} du \quad\brak{Let \frac{x^2}{2}= u}\\
    &= \frac{2}{\sqrt{\pi}} \int_{0}^{\infty} e^{-u} u^{\frac{3}{2}-1} du\\
    &= \frac{2}{\sqrt{\pi}} \Gamma\brak{{\frac{3}{2}}}\\
    &= \frac{1}{\sqrt{\pi}}\Gamma\brak{\frac{1}{2}} \\
    &= 1
\end{align}
where we have used the fact that
\begin{align}
\quad\because \Gamma(n)= (n-1)\Gamma(n-1); \Gamma\brak{\frac{1}{2}}=\sqrt{\pi}
\end{align}
%
Thus, the  variance is
\begin{align}
    \sigma^2 =  E\brak X^2 - E^2\brak X = 1
\end{align}

\item 12 defective pens are accidentally mixed with 132 good ones. It is not possible to just
look at a pen and tell whether or not it is defective. One pen is taken out at random from
this lot. Determine the probability that the pen taken out is a good one.
\\
\solution
%
\begin{align}
    E(X) &= \frac{1}{\sqrt{2\pi}} \int_{-\infty}^{\infty} x e^{-\frac{x^2}{2}}dx\\
    &=0 \quad \brak{ \text{ odd function}}
\end{align}
\begin{align}
    E\brak{X^2}&= \frac{1}{\sqrt{2\pi}}\int_{-\infty}^{\infty} x^2
e^ {-\frac{x^2}{2}} dx \quad \brak{even function}\\
    &= \frac{2}{\sqrt{2\pi}} \int_{0}^{\infty} x^2 e^{-\frac{x^2}{2}} dx\\
    &= \frac{2}{\sqrt{2\pi}}\int_{0}^{\infty}\sqrt{2u}e^{-u} du \quad\brak{Let \frac{x^2}{2}= u}\\
    &= \frac{2}{\sqrt{\pi}} \int_{0}^{\infty} e^{-u} u^{\frac{3}{2}-1} du\\
    &= \frac{2}{\sqrt{\pi}} \Gamma\brak{{\frac{3}{2}}}\\
    &= \frac{1}{\sqrt{\pi}}\Gamma\brak{\frac{1}{2}} \\
    &= 1
\end{align}
where we have used the fact that
\begin{align}
\quad\because \Gamma(n)= (n-1)\Gamma(n-1); \Gamma\brak{\frac{1}{2}}=\sqrt{\pi}
\end{align}
%
Thus, the  variance is
\begin{align}
    \sigma^2 =  E\brak X^2 - E^2\brak X = 1
\end{align}

\item Gopi buys a fish from a shop for his aquarium. The
shopkeeper takes out one fish at random from a
tank containing 5 male fish and 8 female fish (see
Fig. \ref{fig:prob_121}). What is the probability that the fish taken
out is a male fish?
\begin{figure}[!ht]
\centering
\includegraphics[width=\columnwidth]{./prob/figs/woman.eps}
\caption{}
\label{fig:prob_121}
\end{figure}
\\
\solution
%
\begin{align}
    E(X) &= \frac{1}{\sqrt{2\pi}} \int_{-\infty}^{\infty} x e^{-\frac{x^2}{2}}dx\\
    &=0 \quad \brak{ \text{ odd function}}
\end{align}
\begin{align}
    E\brak{X^2}&= \frac{1}{\sqrt{2\pi}}\int_{-\infty}^{\infty} x^2
e^ {-\frac{x^2}{2}} dx \quad \brak{even function}\\
    &= \frac{2}{\sqrt{2\pi}} \int_{0}^{\infty} x^2 e^{-\frac{x^2}{2}} dx\\
    &= \frac{2}{\sqrt{2\pi}}\int_{0}^{\infty}\sqrt{2u}e^{-u} du \quad\brak{Let \frac{x^2}{2}= u}\\
    &= \frac{2}{\sqrt{\pi}} \int_{0}^{\infty} e^{-u} u^{\frac{3}{2}-1} du\\
    &= \frac{2}{\sqrt{\pi}} \Gamma\brak{{\frac{3}{2}}}\\
    &= \frac{1}{\sqrt{\pi}}\Gamma\brak{\frac{1}{2}} \\
    &= 1
\end{align}
where we have used the fact that
\begin{align}
\quad\because \Gamma(n)= (n-1)\Gamma(n-1); \Gamma\brak{\frac{1}{2}}=\sqrt{\pi}
\end{align}
%
Thus, the  variance is
\begin{align}
    \sigma^2 =  E\brak X^2 - E^2\brak X = 1
\end{align}
    
\item A lot consists of 144 ball pens of which 20 are defective and the others are good. Nuri will buy a pen if it is good, but will not buy if it is defective. The shopkeeper draws one pen at random and gives it to her. What is the probability that\\
(i) She will buy it ?\\
(ii) She will not buy it ?
\\
\solution
%
\begin{align}
    E(X) &= \frac{1}{\sqrt{2\pi}} \int_{-\infty}^{\infty} x e^{-\frac{x^2}{2}}dx\\
    &=0 \quad \brak{ \text{ odd function}}
\end{align}
\begin{align}
    E\brak{X^2}&= \frac{1}{\sqrt{2\pi}}\int_{-\infty}^{\infty} x^2
e^ {-\frac{x^2}{2}} dx \quad \brak{even function}\\
    &= \frac{2}{\sqrt{2\pi}} \int_{0}^{\infty} x^2 e^{-\frac{x^2}{2}} dx\\
    &= \frac{2}{\sqrt{2\pi}}\int_{0}^{\infty}\sqrt{2u}e^{-u} du \quad\brak{Let \frac{x^2}{2}= u}\\
    &= \frac{2}{\sqrt{\pi}} \int_{0}^{\infty} e^{-u} u^{\frac{3}{2}-1} du\\
    &= \frac{2}{\sqrt{\pi}} \Gamma\brak{{\frac{3}{2}}}\\
    &= \frac{1}{\sqrt{\pi}}\Gamma\brak{\frac{1}{2}} \\
    &= 1
\end{align}
where we have used the fact that
\begin{align}
\quad\because \Gamma(n)= (n-1)\Gamma(n-1); \Gamma\brak{\frac{1}{2}}=\sqrt{\pi}
\end{align}
%
Thus, the  variance is
\begin{align}
    \sigma^2 =  E\brak X^2 - E^2\brak X = 1
\end{align}


\item A bag contains 5 red balls and some blue balls. If the probability of drawing a blue ball is double that of a red ball, determine the number of blue balls in the bag.
\\
\solution
%
\begin{align}
    E(X) &= \frac{1}{\sqrt{2\pi}} \int_{-\infty}^{\infty} x e^{-\frac{x^2}{2}}dx\\
    &=0 \quad \brak{ \text{ odd function}}
\end{align}
\begin{align}
    E\brak{X^2}&= \frac{1}{\sqrt{2\pi}}\int_{-\infty}^{\infty} x^2
e^ {-\frac{x^2}{2}} dx \quad \brak{even function}\\
    &= \frac{2}{\sqrt{2\pi}} \int_{0}^{\infty} x^2 e^{-\frac{x^2}{2}} dx\\
    &= \frac{2}{\sqrt{2\pi}}\int_{0}^{\infty}\sqrt{2u}e^{-u} du \quad\brak{Let \frac{x^2}{2}= u}\\
    &= \frac{2}{\sqrt{\pi}} \int_{0}^{\infty} e^{-u} u^{\frac{3}{2}-1} du\\
    &= \frac{2}{\sqrt{\pi}} \Gamma\brak{{\frac{3}{2}}}\\
    &= \frac{1}{\sqrt{\pi}}\Gamma\brak{\frac{1}{2}} \\
    &= 1
\end{align}
where we have used the fact that
\begin{align}
\quad\because \Gamma(n)= (n-1)\Gamma(n-1); \Gamma\brak{\frac{1}{2}}=\sqrt{\pi}
\end{align}
%
Thus, the  variance is
\begin{align}
    \sigma^2 =  E\brak X^2 - E^2\brak X = 1
\end{align}

\item A box contains 12 balls out of which x are black. If one ball is drawn at random from the box,what is the probability that it will be a black ball?\\
If 6 more black balls are put in the box, the probability of drawing a black ball is now double of what it was before. Find x.
\\
\solution
%
\begin{align}
    E(X) &= \frac{1}{\sqrt{2\pi}} \int_{-\infty}^{\infty} x e^{-\frac{x^2}{2}}dx\\
    &=0 \quad \brak{ \text{ odd function}}
\end{align}
\begin{align}
    E\brak{X^2}&= \frac{1}{\sqrt{2\pi}}\int_{-\infty}^{\infty} x^2
e^ {-\frac{x^2}{2}} dx \quad \brak{even function}\\
    &= \frac{2}{\sqrt{2\pi}} \int_{0}^{\infty} x^2 e^{-\frac{x^2}{2}} dx\\
    &= \frac{2}{\sqrt{2\pi}}\int_{0}^{\infty}\sqrt{2u}e^{-u} du \quad\brak{Let \frac{x^2}{2}= u}\\
    &= \frac{2}{\sqrt{\pi}} \int_{0}^{\infty} e^{-u} u^{\frac{3}{2}-1} du\\
    &= \frac{2}{\sqrt{\pi}} \Gamma\brak{{\frac{3}{2}}}\\
    &= \frac{1}{\sqrt{\pi}}\Gamma\brak{\frac{1}{2}} \\
    &= 1
\end{align}
where we have used the fact that
\begin{align}
\quad\because \Gamma(n)= (n-1)\Gamma(n-1); \Gamma\brak{\frac{1}{2}}=\sqrt{\pi}
\end{align}
%
Thus, the  variance is
\begin{align}
    \sigma^2 =  E\brak X^2 - E^2\brak X = 1
\end{align}

\item There are 40 students in Class X of a school of whom 25 are girls and 15 are boys. The class teacher has to select one student as a class representative. She writes the name of each student on a separate card, the cards being identical.Then
she puts cards in a bag and stirs them thoroughly. She then draws one card from the
bag. What is the probability that the name written on the card is the name of\\
(i) a girl?\\
(ii) a boy?
\\
\solution 
Let the random variable X = \{0,1\} represent the outcome whether the picked card has a girl's name or a boy's name.  Then
\begin{align}
    \implies \pr{X=0} = \frac{25}{40}\\
    \implies \pr{X=1} = \frac{15}{40}
\end{align}
%
\item A bag contains 3 red balls and 5 black balls. A ball is drawn at random from the bag.
What is the probability that the ball drawn is\\ 
(i) red ? \\
(ii) not red?
Total number of marbles = 3 + 5 = 8 marbles\\
Let $X \in \{0,1\}$ represent the random variable, where 0 represents a red marble, 1 represents a black marble. From the given information, 

\begin{enumerate}
    
\item Probability that the ball taken out will be red = \pr{X=0}
\begin{align}
    \pr{X=0} &= \frac{\text{number of red balls}}{\text{total number of balls}}\\
    \pr{X=0} &= \frac{3}{8} = 0.375
\end{align}
\item Probability that the marble taken out will not be red = \pr{X = 1}\\
Because the complementary of \pr{X=0} is \pr{X=1}\\
We know that the sum of probabilities of every random variable is 1. So,
\begin{align}
    \pr{X = 1}+\pr{X=0}=1
\end{align}
\begin{align}
    \implies \pr{X = 1} &= 1 -\pr{X=0} \\
    &= 1-0.375 \\
    &= 0.625
\end{align}
\begin{align}
    \implies \pr{X = 1} = 1 - 0.375 = 0.625
\end{align}
\end{enumerate}
\end{enumerate}
%\end{document}
    
 
\section{Bayes Rule}
\renewcommand{\theequation}{\theenumi}
\begin{enumerate}[label=\thesection.\arabic*.,ref=\thesection.\theenumi]
\numberwithin{equation}{enumi}


\item Bag I contains 3 red and 4 black balls and Bag II contains 4 red and 5 black balls. One ball is transferred from Bag I to Bag II and then a ball is drawn from Bag II. The ball so drawn is found to be red in colour. Find the probability that the transferred ball is black.\\
\item Suppose we have four boxes A,B,C and D containing coloured marbles as given below:\\
\\$\begin{tabular}{||c c c c||} 
 \hline
 Box & Red & White & Black \\
 \hline
 A & 1 & 6 & 3 \\
 \hline
 B & 6 & 2 & 2 \\
 \hline
 C & 8 & 1 & 1 \\
 \hline
 D & 0 & 6 & 4 \\
 \hline
\end{tabular}$\\
\\One of the boxes has been selected at random and a single marble is drawn from
it. If the marble is red, what is the probability that it was drawn from box A?, box B?,
box C?\\
\solution 
The description of the random variables is available in Table \ref{table:2.2}.
\begin{table}[!ht]
\centering
\begin{tabular}{|c|c|c|c|c|}
\hline
     &A&B&C&D  \\
     \hline
     X&0&1&2&3\\
     \hline
\end{tabular}\\[5pt]
\begin{tabular}{|c|c|c|c|}
\hline
     &Red&White&Black  \\
     \hline
     Y&0&1&2\\
     \hline
\end{tabular}\\[5pt]
\caption{}
\label{table:2.2}
\end{table}
From the given information,
\begin{align}
\pr{X\!=\!0} = \pr{X\!=\!1} &= \pr{X\!=\!2} 
\\
= \pr{X\!=\!3} &= \frac{1}{4}
\\
\pr{Y\!=\!0|X\!=\!0}&=\frac{1}{10}\\
\\\pr{Y\!=\!0|X\!=\!1}&=\frac{6}{10}\\
\\\pr{Y\!=\!0|X\!=\!2}&=\frac{8}{10}\\
\pr{Y\!=\!0|X\!=\!3} &=0
\end{align}
Thus,
\begin{align}
\pr{Y=0}&= \sum_{i=0}^{3}\pr{Y=0|X=i}\pr{X=i}
\\
&=\frac{3}{8}
\end{align}
\begin{enumerate}
\item  
\begin{multline}
Pr(X=0|Y=0)
\\
=\frac{{Pr(Y=0|X=0)}{Pr(X=0)}}{{Pr(Y=0)}}
\\
=\frac{1}{15}
\end{multline}
\item 
\begin{multline}
Pr(X=1|Y=0)
\\
=\frac{{Pr(Y=0|X=1)}{Pr(X=1)}}{{Pr(Y=0)}}
\\
=\frac{2}{5}
\end{multline}
\item 
\begin{multline}
Pr(X=2|Y=0)
\\=\frac{{Pr(Y=0|X=2)}{Pr(X=2)}}{{Pr(Y=0)}}
\\
=\frac{8}{15}
\end{multline}
\end{enumerate}

\item Assume that the chances of a patient having a heart attack is 40$\%$. It is also
assumed that a meditation and yoga course reduce the risk of heart attack by 30$\%$ and prescription of certain drug reduces its chances by 25$\%$. At a time a patient can choose any one of the two options with equal probabilities. It is given that after going through one of the two options the patient selected at random suffers a heart attack. Find the probability that the patient followed a course of meditation and yoga?\\
\solution
Let $H \in \{ 0, 1\}$ denote the random variable of the patient having a heart attack, $A \in \{ 0, 1\} $ denote the random variable of the patient taking a meditation and yoga course, or the patient taking the drug. ($A=0$ if the patient took a meditation and yoga course, and $A=1$ if the patient took the prescription of the drug.)\\
Given that,
\begin{align}
\text{Pr}(H=1) &= 0.4 \\
\text{Pr}(A=0) &= \text{Pr}(A=1)\\
\text{Pr}(H=1|A=0) &= \text{Pr}(H=1)\,(1-0.30)\\
&= 0.28 \\
\text{Pr}(H=1|A=1) &= \text{Pr}(H=1)\,(1-0.25)\\
&= 0.3\\
\end{align}
Therefore, by Bayes' Theorem
\begin{align}
\text{Pr}(A=0|H=1) = \dfrac{\text{Pr}(H=1|A=0)\,\text{Pr}(A=0)}{\sum_{i=0} ^ 1 \text{Pr}(H=1|A=i)\,\text{Pr}(A=i)} \\
\end{align}
We can cancel $\text{Pr}(A=1)$ and $\text{Pr}(A=0)$ from the numerator and denominator as they are given to be equal.\\
\begin{align}
\therefore \text{Pr}(A=0|H=1) &= \dfrac{\text{Pr}(H=1|A=0)}{\text{Pr}(H=1|A=0) + \text{Pr}(H=1|A=1)} \\
&= \dfrac{0.28}{0.28 + 0.3} \\ ~\\[-1em]
&= \dfrac{0.28}{0.58} \\ ~\\[-1em]
&= \dfrac{14}{29} \\ ~\\[-1em]
&\approx 0.48275862069
\end{align}
\item Suppose that 5$\%$ of men and 0.25$\%$ of women have grey hair. A grey haired person is selected at random. What is the probability of this person being male? Assume that there are equal number of males and females.\\
\solution
Let A={0,1} represent the random variable for being male or female and G={0,1} represent having grey hair or not.  Then,\\
  \begin{align}
      P(A=0)&=50\%=\frac{1}{2}\\
      P(A=1)&=50\%=\frac{1}{2}\\
      P(G=1|A=0)&=5\%=\frac{1}{20}\\
      P(G=1|A=1)&=0.25\%=\frac{1}{400}
  \end{align}
  By Bayes rules,\\
  \begin{align}
     P(A=0|G=1)&=\dfrac{P(0)\times P(G=1|0)}{\{\sum}_{i=0}^{1} \pr{G}\\ 
     \because {\sum}_{i=0}^{1}\pr{G}=&P(0)\times P(G=1|0)+P(1)\times P(G=1|1)\\
      P(A=0|G=1)&=\dfrac{\dfrac{1}{2}\times \dfrac{1}{20}}{\dfrac{1}{2} \times \dfrac{1}{20} +\dfrac{1}{2} \times \dfrac{1}{400}}\\
      P(A=0|G=1)&=\dfrac{20}{21}
  \end{align}
  which is the desired probability.
%Let A={0,1} represent the random variable for being male or female and G={0,1} represent having grey hair or not.  Then,\\
  \begin{align}
      P(A=0)&=50\%=\frac{1}{2}\\
      P(A=1)&=50\%=\frac{1}{2}\\
      P(G=1|A=0)&=5\%=\frac{1}{20}\\
      P(G=1|A=1)&=0.25\%=\frac{1}{400}
  \end{align}
  By Bayes rules,\\
  \begin{align}
     P(A=0|G=1)&=\dfrac{P(0)\times P(G=1|0)}{\{\sum}_{i=0}^{1} \pr{G}\\ 
     \because {\sum}_{i=0}^{1}\pr{G}=&P(0)\times P(G=1|0)+P(1)\times P(G=1|1)\\
      P(A=0|G=1)&=\dfrac{\dfrac{1}{2}\times \dfrac{1}{20}}{\dfrac{1}{2} \times \dfrac{1}{20} +\dfrac{1}{2} \times \dfrac{1}{400}}\\
      P(A=0|G=1)&=\dfrac{20}{21}
  \end{align}
  which is the desired probability.

\item A couple has two children,\\
(i) Find the probability that both children are males, if it is known that at least one of the children is male.\\
(ii) Find the probability that both children are females, if it is known that the elder child is a female.\\
\item A manufacturer has three machine operators A, B and C. The first operator A
produces 1$\%$ defective items, where as the other two operators B and C produce 5$\%$ and 7$\%$ defective items respectively. A is on the job for 50$\%$ of the time, B is on the job for 30$\%$ of the time and C is on the job for 20$\%$ of the time. A defective item is produced, what is the probability that it was produced by A?\\
\solution
Let X $\in\{0,1,2\}$ be the random variable denoting that item was produced by operator A when X$=$0, X$=$1 denoting that item was produced by operator B ,X$=$2 denoting that item was produced by operator C, and random variable Y $\in\{0,1\}$ be the random variable deoting that item produced was  defective when Y$=$1.
\begin{align}
	\pr{X=0} &= 0.5 \\ 
    \pr{X=1} &= 0.3 \\ 
    \pr{X=2} &= 0.2 \\ 
	\pr{Y=1/X=0} &= 0.01\\
	\pr{Y=1/X=1} &= 0.05\\ 
	\pr{Y=1/X=2} &= 0.07
	\end{align}
	%
	From conditional probability we say that \\
	\begin{align*}
	\pr{X=0/Y=1} &= \frac{\pr{Y=1/X=0}\pr{X=0}}{\sum_{i=0}^{i=2}\pr{Y=1/X=i}\pr{X=i}}\\
	&=\frac{5}{34}
	\end{align*}


\item A factory has two machines A and B. Past record shows that machine A produced 60$\%$ of the items of output and machine B produced 40$\%$ of the items. Further, 2$\%$ of the items produced by machine A and 1$\%$ produced by machine B were defective. All the items are put into one stockpile and then one item is chosen at random from this and is found to be defective. What is the probability that it was produced by machine B?\\
%
\solution
Let $A \in \cbrak{0,1}$ denote the random variables of an item produced and $D \cbrak{0,1}$ denote it being defective
%
From the given information,
\begin{align}
P(A=0) = 60 {\%} = 0.6\\
P(A=1) = 40 {\%} =0.4
\end{align}
\begin{align}
    P(D=1|A=0) = P(1|0) = 2 {\%} =0.02
   \\P(D=1|A=1) = P(1|1) =  1 {\%} =0.01
\end{align}
By Baye's rule,
\begin{multline}
    P(A=1|D=1) 
    \\
    = \frac{P(1)\times P(1|1)}{P(1)\times P(1|1) + P(0)\times P(1|0)}
    \\P(A=1|D=1) = \frac{0.4\times 0.01}{0.4\times 0.01 + 0.6\times 0.02}
    \\P(A=1|D=1) = 0.25
\end{multline}
The probability that the defective item selected at random is produced by machine B is 25{\%}
\item Two groups are competing for the position on the Board of directors of a corporation. The probabilities that the first and the second groups will win are 0.6 and 0.4 respectively. Further, if the first group wins, the probability of introducing a new product is 0.7 and the corresponding probability is 0.3 if the second group wins. Find the probability that the new product introduced was by the second group.\\
%
\solution
\begin{itemize}
    \item Let $X$ and $Y$ be the input variables which can be referred from the table \ref{bayes/2/8/tab:table1}:-
    \numberwithin{table}{section}
    \begin{table}[ht!]
    \begin{center}
    \begin{tabular}{|l|l|}
    \hline
    \multirow{2}{*}{X} & X=1 : New product is introduced \\ \cline{2-2} 
                       & X=0 : No New product \\ \hline
    \multirow{2}{*}{Y} & Y=1 :  First group wins            \\ \cline{2-2} 
                       & Y=0 : Second group wins \\ \hline
    \end{tabular}
    \end{center}
    \caption{Assumed Variables}
    \label{bayes/2/8/tab:table1}
    \end{table} 
    \item Furthermore, Data given is tabularised in the table \ref{bayes/2/8/tab:table2} :-
    \numberwithin{table}{section}
    \begin{table}[ht!]
    \begin{center}
    \begin{tabular}{|c|c|c|}
    \hline
     &Expression & Value \\
    \hline 
    a.)&Pr(Y=1) & $0.6$ \\ 
    \hline 
    b.)&Pr(Y=0) & $0.4$ \\ 
    \hline 
    c.)&Pr$(X=1|Y=1)$ & $0.7$ \\ 
    \hline 
    d.)&Pr$(X=1|Y=0)$ & $0.3$ \\ 
    \hline 
    \end{tabular}
    \end{center}
    \caption{Data Given}
    \label{bayes/2/8/tab:table2}
    \end{table}
    \item So, the probability that new product is introduced by the second group can be given using \textbf{Baye's Theorem} as:-
    \begin{multline}
      \implies  \text{Pr}(Y=0|X=1)
        \\
       = \frac{\text{Pr}(X=1|Y=0)\text{Pr}(Y=0)}{ \sum_{i=0}^{1}\text{Pr}(X=1|Y=i)\text{Pr}(Y=i)}
        \end{multline}
    \begin{align}
     \implies  \text{Pr}(Y=0|X=1)  &= \frac{0.3\times 0.4}{0.3\times0.4+0.7 \times 0.6 }
        \\
     \implies  \text{Pr}(Y=0|X=1) &= \frac{0.12}{0.54} =\frac{2}{9} 
    \end{align}
    \begin{align}
       \boxed{\therefore \text{Pr}(Y=0|X=1)= \frac{2}{9}}
    \end{align}
\end{itemize}

\item A laboratory blood test is 99$\%$ effective in detecting a certain disease when it is in fact, present. However, the test also yields a false positive result for 0.5$\%$ of the healthy person tested (i.e. if a healthy person is tested, then, with probability 0.005, the test will imply he has the disease). If 0.1 percent of the population actually has the disease, what is the probability that a person has the disease given that his test result is positive?\\
\item A family has two children. What is the probability that both the children are boys given that at least one of them is a boy?\\
%
\solution
X - Random variable for number of boys.
$$X = \{0, 1, 2\}$$
where $n = 2$ and $p = \frac{1}{2}$
\begin{table}[h]
    \centering
    \begin{tabular}{|c|c|}
         \hline
        \textbf{X = x}&\textbf{\pr{X = x}} \\
        \hline    
         {X = 0} & {${^2 C _0} \times {q^2}$}\\
         \hline
         {X = 1} & {${^2 C _1} \times {q} \times {p}$}\\
         \hline
         {X = 2} & {${^2 C _2} \times {p^2}$}\\
         \hline
    \end{tabular}
\end{table}
\\
To find $\pr{X = 2\, | \,X \geq 1}$.
\begin{align}
    \pr{X = 2\, | \,X \geq 1} &= \dfrac{\pr{X = 2}}{\pr{X\geq 1}}\\
    &= \dfrac{\frac{1}{4}}{\frac{3}{4}}\\
    &= \dfrac{1}{3}
\end{align}
\item Ten cards numbered 1 to 10 are placed in a box, mixed up thoroughly and then one card is drawn randomly. If it is known that the number on the drawn card is more than 3, what is the probability that it is an even number?\\
\solution
The set of sample space which contains cards numbered from 1 to 10 be 
\begin{align}
S \in \{1,2,3,4,5,6,7,8,9,10\}
\end{align}
Now probability of picking a random card from ten cards is be $Pr(x)$. Then,
\begin{align}
Pr(x\in S )=\frac{1}{10}
\end{align}
Let the Set of Even numbered cards be $E$ and Set of cards numbered greater than 3 is $A$.
and we know that the set of even numbers from 1 to 10 is \{2,4,6,8,10\} . Then,
\begin{align}
E \in \{2,4,6,8,10\}    
\end{align}
Now here, The Set of numbers greater than 3 
is \{4,5,6,7,8,9,10\}. Then,
\begin{align}
    A \in \{4,5,6,7,8,9,10\}
\end{align}
\begin{align}
    Pr(A)=\frac{No.\; of\; elements\; in\; (A).}{Number\; of\; cards.}
\end{align}
\begin{align}
\label{2.11:eq1}
    Pr(A)=\frac{7}{10}=0.7
\end{align}
Now, the favoured outcomes are set of $EA$ and 
Here the set $EA$ contains the cards which are even and numbered greater than 3.
\begin{align}
    Pr(EA)=\frac{No.\; of\; elements\; in\; E A.}{Number\; of\;cards.}
\end{align}
\begin{align}
   Now , Pr(EA) = Pr(4)+Pr(6)+Pr(8)+Pr(10)
\end{align}
\begin{align}
\label{2.11:eq2}
     Pr(EA) =\frac{4}{10}=0.4 
\end{align}
The probability that the card drawn is even number which is greater than 3 is
\begin{align}
    Pr(E|A)=\frac{Pr(EA)}{Pr(A)}
\end{align}
Now,using (\ref{2.11:eq1}) and (\ref{2.11:eq2}). 
\begin{align}
  (\because Pr(EA)=0.4\; and \; Pr(A)=0.7)  
\end{align}
\begin{align}
And, Pr(E|A)=\frac{Pr(EA)}{Pr(A)}
\end{align}
\begin{align}
=\frac{0.4}{0.7}
\end{align}
\begin{align}
\therefore Pr(E|A)=\frac{4}{7}    
\end{align}
\begin{align}
\therefore Pr(E|A)=0.5714285714285714    
\end{align}

\item In a school, there are 1000 students, out of which 430 are girls. It is known that out of 430,  10 percentage of the girls study in class XII. What is the probability that a student chosen randomly studies in Class XII given that the chosen student is a girl?\\
\solution
Total number of students : 1000 \\
Total number of girls : 430  \\
Total number of girls in Class XII : 10 \% of total girls 
\begin{align}
=\frac{10}{100}\times 430 =43
\end{align}
Let X $\in$ $\{0,1\}$ be the random variable such that 1 represents girl, 0 represents boy.
\begin{table}[ht]
\caption{Probability distribution for values of X}
\label{2.12:table:0}
\begin{center}
\begin{tabular}{|c|c|}
    \hline
    X & P(X) \\
    \hline
    1 & 430/1000\\
    \hline
    0 & 570/100\\
    \hline
    \end{tabular} 
\end{center}   
\end{table}

Let Y $\in$ $\{0,1\}$ be the random variable such that 1 represents chosen student is in Class XII, 0 represents chosen student is not in Class XII.

\begin{table}[ht]
\caption{Probability distribution for values of Y}
\label{2.12:table:1}
\begin{center}
    \begin{tabular}{|c|c|}
    \hline
    Y & P(Y)\\
    \hline
    1 & $\frac{1}{2}$ \\
    \hline
    0 & $\frac{1}{2}$\\
    \hline
    \end{tabular}
\end{center} 
\end{table}
We require P(Y=1 $\vert$ X=1) (using Baye's theorem)
\begin{align}
=\dfrac{P(X=1\vert Y=1)\cdot P(Y=1)}{\sum_{i=0}^{1} P(X=1\vert Y=i)\cdot P(Y=i)}
\end{align}
\begin{center} 
=$\frac{P(X=1 \vert Y=1)P\brak(Y=1)}{\splitfrac{P(X=1\vert Y=0)P(Y=0)}{+P(X=1\vert Y=1)P(Y=1)}}$
\end{center}
\begin{table}[ht]
\caption{Probability for different values of X,Y}
\label{2.12:table:2}
\begin{center}
    \begin{tabular}{|c|c|c|}
    \hline
    Probability & Chosen Student & Value\\
    \hline
    $P(X=1\vert Y=0)$ & girl not in Class XII & $\frac{387}{1000}$\\
    \hline
    $ P(X=1\vert Y=1)$ & girl in Class XII & $\frac{43}{1000}$ \\
    \hline
    \end{tabular}
\end{center}    
\end{table} 
\begin{align}
&=\left[\dfrac{\brak( \frac{43}{1000}\times \frac{1}{2})}{\brak( \frac{387}{1000})\times \frac{1}{2}) + \brak( \frac{43}{1000}\times \frac{1}{2})}\right]\\
&=\dfrac{1}{10}\\ 
&=0.1
\end{align}

\item A die is thrown three times. Events A and B are defined as below:\\
A : 4 on the third throw.\\
B : 6 on the first and 5 on the second throw.\\
Find the probability of A given that B has already occurred?\\
\solution
Let $X_i \in \{1,2,3,4,5,6\}$ where $i=1,2,3$
be the random variables representing the outcomes of throwing a die three times.\\
\begin{enumerate}
\item Probability of event A happening=Probability of $X_3=4$
\begin{align}
    \Pr{(A)}=\Pr{(X_3=4)}
\end{align}
Since all the outcomes are equally likely their probabilities are same \\
so
\begin{align}
    \Pr{(A)}=\Pr{(X_3=4)}=\frac{1}{6}
    \label{2.13:eq:0.0.2}
\end{align}
\item Probability of event B happening=Probability of $X_1=6,X_2=5$.\\
so
\begin{align}
    \Pr{(B)}=\Pr{(X_1=6,X_2=5)}
\end{align}
Random variable $X_1$ depends on first throw of die and random variable $X_2$ depends on second throw of die so $X_1$ and $X_2$ are independent.\\
so 
\begin{align}
\begin{split}
    \Pr{(X_1=6,X_2=5)} &=\Pr{(X_1=6)} \Pr{(X_2=5)}\\
   &=\frac{1}{6}\times \frac{1}{6}=\frac{1}{36}
   \end{split}
\end{align}
\begin{align}
    \Pr{(B)}=\Pr{(X_1=6,X_2=5)}=\frac{1}{36}
    \label{2.13:eq:0.0.5}
\end{align}
Also A,B are also independent events \\
therefore  from \eqref{2.13:eq:0.0.2} and \eqref{2.13:eq:0.0.5}
\begin{align}
    \Pr{(AB)} = \Pr{(A)} \Pr{(B)}
     =\frac{1}{6} \times \frac{1}{36}
    \end{align}
    \begin{align}
 \implies  \Pr{(AB)}=\frac{1}{216}\label{2.13:eq:0.0.7}
\end{align}
Since we have to find probability of A given that B has already happened.\\
so $\Pr{(A|B)}$\\
\item By formula of conditional probability
\begin{align}
    \Pr{(A|B)}=\frac{\Pr{(AB)}}{\Pr{(B)}}
    \end{align}
    From \eqref{2.13:eq:0.0.5} and \eqref{2.13:eq:0.0.7} 
    \begin{align}  
    \implies
    \Pr{(A|B)}=\frac{\frac{1}{216}}{\frac{1}{36}}\\
    \implies
    \Pr{(A|B)}=\frac{1}{6}
    \end{align}
    So the probability of A given that B has already happened $=\Pr{(A|B)}=\frac{1}{6}$
\end{enumerate}

\item A die is thrown twice and the sum of the numbers appearing is observed to be 6. What is the conditional probability that the number 4 has appeared at least once?\\
\solution
Let $A \in \cbrak{2,3,4,5,6,7,8,9,10,11,12}$ be a random variable representing the sum of outcomes when a die is thrown twice.\\
Let $B \in \cbrak{0,1,2}$  be a random variable that represents the number of times 4 occurs in two throws.\\
We need the conditional probability of event $\brak{B \geq 1}$ given that \brak{A=6} has occurred.\\ 
\begin{equation}
    \Pr{\brak{\brak{B \geq 1}|\brak{A=6}}}
    =\frac{\Pr{((A=6)\cap(B \geq 1))}}{\Pr{\brak{A=6}}}\label{eq:0.0.1}
\end{equation}
We have that,\\
\begin{equation}
\Pr{(A=n)}=
\begin{cases}
    0 & n \leq 1\\
    \frac{n-1}{36} & 2\leq n \leq 7\\
    \frac{13-n}{36} & 8 \leq n \leq 12\\
    0 & n \geq 13
\end{cases}
\end{equation}
Therefore using equation (0.0.2) we can write that,
\begin{equation}
    \Pr{(A=6)}=\frac{5}{36}\label{eq:0.0.3}
\end{equation}
From binomial distribution we can write ,
\begin{align}
    \Pr{(B \geq 1)}&=\Pr{(B =1)}+\Pr{(B=2)} \\ &= \binom{2}{1} \brak {\frac{1}{6}}\brak {\frac{5}{6}}+\binom{2}{2} \brak {\frac{1}{6}}^2\\
    &=\frac{11}{36}
\end{align}

The event $((A=6) \cap (B\geq 1))$ is such that the sum should be six 6 with atleast one 4.\\
There are only two possible cases \{4,2\},\{2,4\} out of 36 possible cases.\\
Hence,\\
\begin{equation}
    \Pr{((A=6) \cap (B\geq 1))}=\frac{2}{36}.\label{eq:0.0.7}
\end{equation}
Substituting equations \eqref{eq:0.0.3},\eqref{eq:0.0.7} in \eqref{eq:0.0.1} , we get\\
\begin{equation}
\begin{split}
\Pr((B \geq 1)|(A=6
))&=\frac{\frac{2}{36}}{\frac{5}{36}}\\
&=\frac{2}{5}.
\end{split}
\end{equation}
Hence the probability of occurring atleast one 4 when the sum of the numbers is 6 when a die is thrown twice is $\frac{2}{5}$. 

\item Consider the experiment of tossing a coin. If the coin shows head, toss it again but if it shows tail, then throw a die. Find the conditional probability of the event that "the die shows a number greater than 4" given that "there is at least one tail".\\
\solution
Given that a coin is tossed. If coin shows head, it is tossed again. If it shows tail, then a die is thrown.
\\Let X $\in$ $\{0,1\}$ be the random variable such that 1 represents occurrence of tail,0 represents occurrence of head when coin is tossed.
\begin{table}[ht]
\caption{Probability distribution for values of X}
\begin{center}
    \begin{tabular}{|c|c|}
    \hline
    X & P(X)\\
    \hline
    1 & $\frac{1}{2}$ \\
    \hline
    0 & $\frac{1}{2}$\\
    \hline
    \end{tabular}
\end{center} 
\end{table}
\\Let Y denotes random variable for the getting a number on the die thrown, then the probability distribution is
\begin{table}[ht]
\caption{Probability distribution for values of Y}
\begin{center}
    \begin{tabular}{|c|c|c|c|c|c|c|}
    \hline
    Y & 1 & 2 & 3 & 4 & 5 & 6 \\
    \hline
    P(Y) & $\frac{1}{6}$ & $\frac{1}{6}$ & $\frac{1}{6}$ & $\frac{1}{6}$ & $\frac{1}{6}$ & $\frac{1}{6}$  \\
    \hline
    \end{tabular}
\end{center} 
\end{table}
\begin{align}
  \text{Pr}(X=1)
=& \sum_{i=1}^{6}\text{Pr}(X=1,Y=i)+\text{Pr}(X=0,X=1)\\
=&\frac{3}{4}
\end{align}
\begin{align}
 \text{Pr}(X=1,Y>4)
=&\text{Pr}(X=1,Y=5)+\text{Pr}(X=1,Y=6)\\
=&\frac{1}{6}  
\end{align}
\begin{align}
\text{Pr}(Y>4|X=1)= &\frac{\text{Pr}(Y>4,X=1)}{\text{Pr}(X=1)}\\
 = &\frac{2}{9}
\end{align}


\item An urn contains 10 black and 5 white balls. Two balls are drawn from the urn one after the other without replacement. What is the probability that both drawn balls are black?\\

\item Three cards are drawn successively, without replacement from a pack of 52 well shuffled cards. What is the probability that first two cards are kings and the third card drawn is an ace?\\

\item A man is known to speak truth 3 out of 4 times. He throws a die and reports that it is a six. Find the probability that it is actually a six.\\

\item A person has undertaken a construction job. The probabilities are 0.65 that there will be strike, 0.80 that the construction job will be completed on time if there is no strike, and 0.32 that the construction job will be completed on time if there is a strike. Determine the probability that the construction job will be completed on time.\\
\solution
\input{./solutions/20-30/chapters/prob/examples/docq25.tex}

\item Bag I contains 3 red and 4 black balls while another Bag II contains 5 red and 6 black balls. One ball is drawn at random from one of the bags and it is found to be red. Find the probability that it was drawn from Bag II.\\
\solution
\input{./solutions/20-30/chapters/prob/examples/docq26.tex}

\item Given three identical boxes I, II and III, each containing two coins. In box I, both coins are gold coins, in box II, both are silver coins and in the box III, there is one gold and one silver coin. A person chooses a box at random and takes out a coin. If the coin is of gold, what is the probability that the other coin in the box is also of gold?\\
\solution
\input{./solutions/20-30/chapters/prob/examples/docq27.tex}

\item Suppose that the reliability of a HIV test is specified as follows: Of people having HIV, 90$\%$ of the test detect the disease but 10$\%$ go undetected. Of people free of HIV, 99$\%$ of the test are judged HIV –ve but 1$\%$ are diagnosed as showing HIV +ve. From a large population of which only 0.1$\%$ have HIV, one person is selected at random, given the HIV test, and the pathologist reports him/her as HIV +ve. What is the probability that the person actually has HIV?\\
\solution
\input{./solutions/20-30/chapters/prob/examples/docq28.tex}

\item In a factory which manufactures bolts, machines A, B and C manufacture respectively 25$\%$, 35$\%$ and 40$\%$ of the bolts. Of their outputs, 5, 4 and 2 percent are respectively defective bolts. A bolt is drawn at random from the product and is found to be defective. What is the probability that it is manufactured by the machine B?\\
\solution
\input{./solutions/20-30/chapters/prob/examples/docq29.tex}

\item A doctor is to visit a patient. From the past experience, it is known that the probabilities that he will come by train, bus, scooter or by other means of transport are respectively $\frac{3}{10},\frac{1}{5},\frac{1}{10}$ and $\frac{2}{5}.$ The probabilities that he will be late are $\frac{1}{4},\frac{1}{3}$ and $\frac{1}{12},$ if he comes by train, bus and scooter respectively, but if he comes by other means of transport, then he will not be late. When he arrives, he is late. What is the probability that he comes by train?\\
\solution
\input{./solutions/20-30/chapters/prob/examples/docq30.tex}

\item Coloured balls are distributed in four boxes as shown in Table \ref{table:1.43_boxes}

\begin{table}[ht!]
\centering
\input{./solutions/20-10/prob/tables/boxes43.tex}
\caption{Distribution of the balls in the boxes}
\label{table:1.43_boxes}
\end{table}
A box is selected at random and then a ball is randomly drawn from the selected box. The colour of the ball is black, what is the probability that ball drawn is from the box III?
\\
\solution
%
\begin{align}
    E(X) &= \frac{1}{\sqrt{2\pi}} \int_{-\infty}^{\infty} x e^{-\frac{x^2}{2}}dx\\
    &=0 \quad \brak{ \text{ odd function}}
\end{align}
\begin{align}
    E\brak{X^2}&= \frac{1}{\sqrt{2\pi}}\int_{-\infty}^{\infty} x^2
e^ {-\frac{x^2}{2}} dx \quad \brak{even function}\\
    &= \frac{2}{\sqrt{2\pi}} \int_{0}^{\infty} x^2 e^{-\frac{x^2}{2}} dx\\
    &= \frac{2}{\sqrt{2\pi}}\int_{0}^{\infty}\sqrt{2u}e^{-u} du \quad\brak{Let \frac{x^2}{2}= u}\\
    &= \frac{2}{\sqrt{\pi}} \int_{0}^{\infty} e^{-u} u^{\frac{3}{2}-1} du\\
    &= \frac{2}{\sqrt{\pi}} \Gamma\brak{{\frac{3}{2}}}\\
    &= \frac{1}{\sqrt{\pi}}\Gamma\brak{\frac{1}{2}} \\
    &= 1
\end{align}
where we have used the fact that
\begin{align}
\quad\because \Gamma(n)= (n-1)\Gamma(n-1); \Gamma\brak{\frac{1}{2}}=\sqrt{\pi}
\end{align}
%
Thus, the  variance is
\begin{align}
    \sigma^2 =  E\brak X^2 - E^2\brak X = 1
\end{align}


\item If a machine is correctly set up, it produces 90$\%$ acceptable items. If it is
incorrectly set up, it produces only 40$\%$ acceptable items. Past experience shows that
80$\%$ of the set ups are correctly done. If after a certain set up, the machine produces 2 acceptable items, find the probability that the machine is correctly setup.
\\
\solution
%
\begin{align}
    E(X) &= \frac{1}{\sqrt{2\pi}} \int_{-\infty}^{\infty} x e^{-\frac{x^2}{2}}dx\\
    &=0 \quad \brak{ \text{ odd function}}
\end{align}
\begin{align}
    E\brak{X^2}&= \frac{1}{\sqrt{2\pi}}\int_{-\infty}^{\infty} x^2
e^ {-\frac{x^2}{2}} dx \quad \brak{even function}\\
    &= \frac{2}{\sqrt{2\pi}} \int_{0}^{\infty} x^2 e^{-\frac{x^2}{2}} dx\\
    &= \frac{2}{\sqrt{2\pi}}\int_{0}^{\infty}\sqrt{2u}e^{-u} du \quad\brak{Let \frac{x^2}{2}= u}\\
    &= \frac{2}{\sqrt{\pi}} \int_{0}^{\infty} e^{-u} u^{\frac{3}{2}-1} du\\
    &= \frac{2}{\sqrt{\pi}} \Gamma\brak{{\frac{3}{2}}}\\
    &= \frac{1}{\sqrt{\pi}}\Gamma\brak{\frac{1}{2}} \\
    &= 1
\end{align}
where we have used the fact that
\begin{align}
\quad\because \Gamma(n)= (n-1)\Gamma(n-1); \Gamma\brak{\frac{1}{2}}=\sqrt{\pi}
\end{align}
%
Thus, the  variance is
\begin{align}
    \sigma^2 =  E\brak X^2 - E^2\brak X = 1
\end{align}


\item A bag contains a red ball, a blue ball and a yellow ball, all the balls being
of the same size.Kritika takes out a ball from the bag without looking into it. What is the probability that she takes out the
(i) yellow ball? \\
(ii) red ball?\\
(iii) blue ball?
\\
\solution
%
\begin{align}
    E(X) &= \frac{1}{\sqrt{2\pi}} \int_{-\infty}^{\infty} x e^{-\frac{x^2}{2}}dx\\
    &=0 \quad \brak{ \text{ odd function}}
\end{align}
\begin{align}
    E\brak{X^2}&= \frac{1}{\sqrt{2\pi}}\int_{-\infty}^{\infty} x^2
e^ {-\frac{x^2}{2}} dx \quad \brak{even function}\\
    &= \frac{2}{\sqrt{2\pi}} \int_{0}^{\infty} x^2 e^{-\frac{x^2}{2}} dx\\
    &= \frac{2}{\sqrt{2\pi}}\int_{0}^{\infty}\sqrt{2u}e^{-u} du \quad\brak{Let \frac{x^2}{2}= u}\\
    &= \frac{2}{\sqrt{\pi}} \int_{0}^{\infty} e^{-u} u^{\frac{3}{2}-1} du\\
    &= \frac{2}{\sqrt{\pi}} \Gamma\brak{{\frac{3}{2}}}\\
    &= \frac{1}{\sqrt{\pi}}\Gamma\brak{\frac{1}{2}} \\
    &= 1
\end{align}
where we have used the fact that
\begin{align}
\quad\because \Gamma(n)= (n-1)\Gamma(n-1); \Gamma\brak{\frac{1}{2}}=\sqrt{\pi}
\end{align}
%
Thus, the  variance is
\begin{align}
    \sigma^2 =  E\brak X^2 - E^2\brak X = 1
\end{align}

\item An urn contains 5 red and 5 black balls. A ball is drawn at random, its colour is noted and is returned to the urn. Moreover, 2 additional balls of the colour drawn are put in the urn and then a ball is drawn at random. What is the probability that the second ball is red?\\
\solution
%
\begin{align}
    E(X) &= \frac{1}{\sqrt{2\pi}} \int_{-\infty}^{\infty} x e^{-\frac{x^2}{2}}dx\\
    &=0 \quad \brak{ \text{ odd function}}
\end{align}
\begin{align}
    E\brak{X^2}&= \frac{1}{\sqrt{2\pi}}\int_{-\infty}^{\infty} x^2
e^ {-\frac{x^2}{2}} dx \quad \brak{even function}\\
    &= \frac{2}{\sqrt{2\pi}} \int_{0}^{\infty} x^2 e^{-\frac{x^2}{2}} dx\\
    &= \frac{2}{\sqrt{2\pi}}\int_{0}^{\infty}\sqrt{2u}e^{-u} du \quad\brak{Let \frac{x^2}{2}= u}\\
    &= \frac{2}{\sqrt{\pi}} \int_{0}^{\infty} e^{-u} u^{\frac{3}{2}-1} du\\
    &= \frac{2}{\sqrt{\pi}} \Gamma\brak{{\frac{3}{2}}}\\
    &= \frac{1}{\sqrt{\pi}}\Gamma\brak{\frac{1}{2}} \\
    &= 1
\end{align}
where we have used the fact that
\begin{align}
\quad\because \Gamma(n)= (n-1)\Gamma(n-1); \Gamma\brak{\frac{1}{2}}=\sqrt{\pi}
\end{align}
%
Thus, the  variance is
\begin{align}
    \sigma^2 =  E\brak X^2 - E^2\brak X = 1
\end{align}


\item A bag contains 4 red and 4 black balls, another bag contains 2 red and 6 black balls. One of the two bags is selected at random and a ball is drawn from the bag which is found to be red. Find the probability that the ball is drawn from the first bag.\\
\solution
%
\begin{align}
    E(X) &= \frac{1}{\sqrt{2\pi}} \int_{-\infty}^{\infty} x e^{-\frac{x^2}{2}}dx\\
    &=0 \quad \brak{ \text{ odd function}}
\end{align}
\begin{align}
    E\brak{X^2}&= \frac{1}{\sqrt{2\pi}}\int_{-\infty}^{\infty} x^2
e^ {-\frac{x^2}{2}} dx \quad \brak{even function}\\
    &= \frac{2}{\sqrt{2\pi}} \int_{0}^{\infty} x^2 e^{-\frac{x^2}{2}} dx\\
    &= \frac{2}{\sqrt{2\pi}}\int_{0}^{\infty}\sqrt{2u}e^{-u} du \quad\brak{Let \frac{x^2}{2}= u}\\
    &= \frac{2}{\sqrt{\pi}} \int_{0}^{\infty} e^{-u} u^{\frac{3}{2}-1} du\\
    &= \frac{2}{\sqrt{\pi}} \Gamma\brak{{\frac{3}{2}}}\\
    &= \frac{1}{\sqrt{\pi}}\Gamma\brak{\frac{1}{2}} \\
    &= 1
\end{align}
where we have used the fact that
\begin{align}
\quad\because \Gamma(n)= (n-1)\Gamma(n-1); \Gamma\brak{\frac{1}{2}}=\sqrt{\pi}
\end{align}
%
Thus, the  variance is
\begin{align}
    \sigma^2 =  E\brak X^2 - E^2\brak X = 1
\end{align}


\item Of the students in a college, it is known that 60$\%$ reside in hostel and 40$\%$ are day scholars (not residing in hostel). Previous year results report that 30$\%$ of all students who reside in hostel attain A grade and 20$\%$ of day scholars attain A grade in their annual examination. At the end of the year, one student is chosen at random from the college and he has an A grade, what is the probability that the student is a hostelier?\\
\solution
%
\begin{align}
    E(X) &= \frac{1}{\sqrt{2\pi}} \int_{-\infty}^{\infty} x e^{-\frac{x^2}{2}}dx\\
    &=0 \quad \brak{ \text{ odd function}}
\end{align}
\begin{align}
    E\brak{X^2}&= \frac{1}{\sqrt{2\pi}}\int_{-\infty}^{\infty} x^2
e^ {-\frac{x^2}{2}} dx \quad \brak{even function}\\
    &= \frac{2}{\sqrt{2\pi}} \int_{0}^{\infty} x^2 e^{-\frac{x^2}{2}} dx\\
    &= \frac{2}{\sqrt{2\pi}}\int_{0}^{\infty}\sqrt{2u}e^{-u} du \quad\brak{Let \frac{x^2}{2}= u}\\
    &= \frac{2}{\sqrt{\pi}} \int_{0}^{\infty} e^{-u} u^{\frac{3}{2}-1} du\\
    &= \frac{2}{\sqrt{\pi}} \Gamma\brak{{\frac{3}{2}}}\\
    &= \frac{1}{\sqrt{\pi}}\Gamma\brak{\frac{1}{2}} \\
    &= 1
\end{align}
where we have used the fact that
\begin{align}
\quad\because \Gamma(n)= (n-1)\Gamma(n-1); \Gamma\brak{\frac{1}{2}}=\sqrt{\pi}
\end{align}
%
Thus, the  variance is
\begin{align}
    \sigma^2 =  E\brak X^2 - E^2\brak X = 1
\end{align}


\item In answering a question on a multiple choice test, a student either knows the
answer or guesses. Let $\frac{3}{4}$ be the probability that he knows the answer and $\frac{1}{4}$ be the probability that he guesses. Assuming that a student who guesses at the answer will be correct with probability $\frac{1}{4}$. What is the probability that the student knows the answer given that he answered it correctly?\\
\solution
%
\begin{align}
    E(X) &= \frac{1}{\sqrt{2\pi}} \int_{-\infty}^{\infty} x e^{-\frac{x^2}{2}}dx\\
    &=0 \quad \brak{ \text{ odd function}}
\end{align}
\begin{align}
    E\brak{X^2}&= \frac{1}{\sqrt{2\pi}}\int_{-\infty}^{\infty} x^2
e^ {-\frac{x^2}{2}} dx \quad \brak{even function}\\
    &= \frac{2}{\sqrt{2\pi}} \int_{0}^{\infty} x^2 e^{-\frac{x^2}{2}} dx\\
    &= \frac{2}{\sqrt{2\pi}}\int_{0}^{\infty}\sqrt{2u}e^{-u} du \quad\brak{Let \frac{x^2}{2}= u}\\
    &= \frac{2}{\sqrt{\pi}} \int_{0}^{\infty} e^{-u} u^{\frac{3}{2}-1} du\\
    &= \frac{2}{\sqrt{\pi}} \Gamma\brak{{\frac{3}{2}}}\\
    &= \frac{1}{\sqrt{\pi}}\Gamma\brak{\frac{1}{2}} \\
    &= 1
\end{align}
where we have used the fact that
\begin{align}
\quad\because \Gamma(n)= (n-1)\Gamma(n-1); \Gamma\brak{\frac{1}{2}}=\sqrt{\pi}
\end{align}
%
Thus, the  variance is
\begin{align}
    \sigma^2 =  E\brak X^2 - E^2\brak X = 1
\end{align}


\end{enumerate}
 
\section{Binomial Distribution}
% \renewcommand{\theequation}{\theenumi}
% \begin{enumerate}[label=\arabic*.,ref=\thesubsection.\theenumi]
% \numberwithin{equation}{enumi}

\renewcommand{\theequation}{\theenumi}
\begin{enumerate}[label=\thesection.\arabic*.,ref=\thesection.\theenumi]
\numberwithin{equation}{enumi}


\item How many times must a man toss a fair coin so that the probability of having at least one head is more than 90$\%$?\\
\item An experiment succeeds twice as often as it fails. Find the probability that in the next six trials, there will be at least 4 successes.\\
\solution


As per question,
\begin{align}
    p=2(1-p) \label{eq:2.0.1}\\
    \implies p=2/3\label{eq:2.0.2}
    \end{align}
For a binomial distribution,
\begin {align}
    \pr{X=k}= \comb{n}{k} p^k \brak{1-p}^{n-k}\label{eq:2.0.3}
    \end{align}
For the given question,
\begin{table}[h]
\begin{center}
\begin{tabular}{|c|c|c|}
\hline
 \textbf{Variable} & $n$ & $p$\\
 \hline
 \textbf{Value} & 6 & 2/3\\
 \hline
\end{tabular}
\caption{Value of variables}
\label{Tab 1}
\end{center}
\end{table}
From \eqref{eq:2.0.3} we have,
\begin{align}
\pr{X\geq4}&=\sum_{i=4}^{6}\comb{6}{i} p^i\brak{1-p}^{6-i}
\label{eq: 2.0.4}\\
&=\frac{240}{729}+\frac{192}{729}+\frac{64}{729}
\label{eq: 2.0.5}\\
&=\frac{496}{729}
\label{eq:2.0.6}
\end{align}

\item Suppose X has a binomial distribution . Show that X = 3 is the most likely outcome.\\
\solution
Let X be a binomial random variable which has probability  $p=\frac{1}{2}$ 
 \begin{align}
p(X)=\frac{1}{2} \label{3.3:1}
 \end{align}
Given number of times event(X) is\\
 performed$(n)=6$
 \begin{align}
 n(x)=6 \label{3.3:2}
 \end{align}
Given probability of event$(p)= \frac{1}{2}$\\
Probability that event(X) does not \\occur is
$(1-p)=1-\frac{1}{2}=\frac{1}{2} $
 \begin{align}
1-p(x)=\frac{1}{2}\label{3.3:3}
 \end{align}
We know that binomial probability
\begin{align}
\pr{X=k} = \comb{n}{k} p^k({1-p})^{n-k}  \label{3.3:4} 
\end{align}
For $\pr{X=k}$ to be most likely outcome(highest probability),
 $\pr{X=k}$  should be maximum ,where\\
$ k=\{0,1,2,3,4,5,6\}$\\
To find maximum of  $\pr{X=k}$  ,let us  apply \\logarithm on both sides for equation \eqref{3.3:4} and then diffenrentiate it with respect
to $p$.
\begin{align}
\log  \pr{X=k} &=\log  \comb{n}{k} \times p^k\times ({1-p})^{n-k} \\
&=\log \comb{n}{k} +k \times \log p \notag \\
 &+(n-k)\times log (1-p) \label{3.3:5}
\end{align}
Differentaiate eq \eqref{3.3:5} with respect to p

\begin{align}
\frac{\mathrm{d} \log \pr{X=k} }{\mathrm{d} p}&=\frac{\mathrm{d} \log  \comb{n}{k} }{\mathrm{d}p}+k\times  \frac{\mathrm{d}\log p}{\mathrm{d} p} \\
     &+(n-k) \times \frac{\mathrm{d}\log (1-p)}{\mathrm{d} p} \notag \\
     &=0+\frac{k}{p}-\frac{n-k}{ 1-p} \label{3.3:6}
\end{align}
To find maximum ,substitute $\frac{\mathrm{d} \log  \pr{X=k} }{\mathrm{d} p}=0$ in \eqref{3.3:6}
\begin{align}
\frac{k}{ p}&=\frac{n-k}{1-p}  \\
\frac{ n-k}{k}&=\frac{1-p}{ p}  \\
\frac{ n}{k}-1&=\frac{1}{ p}-1  \\
\frac{n}{ k}&=\frac{ 1}{t p}  \\
k&=n\times p \label{3.3:7}
\end{align}
substituting $n=6,p=1-p=\frac{1}{2}$ in \eqref{3.3:7}\\
We get $k=6\times \frac{1}{2}=3$
$\pr{X=3}$ is maximum,\\
$\therefore\pr{X=3}$ is most likely \\
outcome.\\
(Hint : P(X = 3) is the maximum among all P($x_i$), $x_i$= 0,1,2,3,4,5,6)\\
\item The probability that a bulb produced by a factory will fuse after 150 days of use
is 0.05. Find the probability that out of 5 such bulbs\\
(i) none\\
(ii) not more than one\\
(iii) more than one\\
(iv) at least one\\
will fuse after 150 days of use.\\
\solution
Let X be random variable which denoting number of bulbs fuses after 150 days of use,among the 5 bulbs.Then by Binomial Distribution.
\begin{align}
    \pr{X = r} = {n \choose r} p^r q^{n-r}\\
    \pr{X \geq k} = \sum_{r = k}^n {n \choose r}p^r q^{n-r}\\
    \pr{X \leq k} = \sum_{r = 0}^k {n \choose r}p^r q^{n-r}\\
    \pr{X>k} = \sum_{r=k+1}^{n}{n\choose r}p^r q^{n-r}\\
    n = 5,\quad p = 0.05,\quad q = 0.95
\end{align}

\begin{table}[h!]
\resizebox{\columnwidth}{!}{%
  \begin{tabular}{|c|c|c|c|c|}
    \hline
    n &  5 & 5 & 5& 5\\
    \hline
    Condition & \pr{X = 0} & \pr{X\leq1} & \pr{X > 1} & \pr{X \geq1}\\
    \hline
    Value & 0.77378 & 0.97740 & 0.02259 & 0.22621\\
    \hline
    Case & $(i)$ & $(ii)$ & $(iii)$ & $(iv)$\\
    \hline
  \end{tabular}%
} 
  \caption{ Probability Vs Condition}
  \label{tab:label1_test}
\end{table}  

\item Find the mean number of heads in three tosses of a fair coin.\\
\solution
By observing we can get 4 cases which are 0 heads, 1 heads, 2 heads, 3 heads respectively when 3 fair coins are tossed simultaneously.\\
Let $X_{i}$ be the number of heads in \textit{i}th case .\\
So we can get the probability of number of heads each case  as
\begin{align}
   \pr{X = k} = 
  \begin{cases}
    {^3 C_k}\brak{\frac{1}{2}}^{k}\brak{1-\frac{1}{2}}^{3-k} & 0 \leq k \leq 3\\
    0 & otherwise
  \end{cases}
 \end{align}
 \begin{align}\label{3.5equation-2}
   \pr{X = k} = 
   \begin{cases}
    {^3 C_k}\brak{\frac{1}{2}}^{3} &  0 \leq k \leq 3  \\
    0 & otherwise
  \end{cases}
\end{align}
\\The probability distribution table is \\
\begin{table}[h!]\label{3.5table}
\centering
\begin{tabular}{||c||c||c||c||c||}
\hline\hline
     \textit{i} & 1 & 2 & 3 & 4  \\
     \hline\hline
     $X_{i}$ & 0 & 1 & 2 & 3 \\
     \hline\hline
     $\pr{X = X_{i}}$ & $\frac{1}{8}$ & $\frac{3}{8}$ & $\frac{3}{8}$ & $\frac{1}{8}$\\[1ex]
     \hline\hline
\end{tabular}
\caption{Probability distribution table}
\end{table}
\\Hence by substituing values of \textit{n} = 3 and \textit{p} = $\frac{1}{2}$, we get 
\begin{align}
    E(X) &= np\\
         &= 3 \times \frac{1}{2} \\
         &= \frac{3}{2} \\
         &= 1.5
\end{align}

\item Find the probability distribution of\\
(i) number of heads in two tosses of a coin.\\
(ii) number of tails in the simultaneous tosses of three coins.\\
(iii) number of heads in four tosses of a coin.\\
\solution

\begin{align}\label{3.6:eq1}
   \pr{X=k} =
  \begin{cases*}
    {^n C_k}p^{k}(1-p)^{n-k} & $0 \leq k \leq n$\\
      0 & otherwise
  \end{cases*}
\end{align}

Table \eqref{3.6:Table1} presents the solution to each case.


\begin{table}[h!]
\centering
\caption{Table of Probability distribution of different cases}
\resizebox{.5\textwidth}{!}{
  \begin{tabular}{|c|m{1cm}|m{2cm}|c|c|c|c|c|}
  \hline
    Case &  n (no. of coins) & k (no. of required outcomes) & 0 & 1 & 2 & 3 & 4\\
    \hline
    $(i)$ & 2 & $\pr{X=k}={^2 C_k}\brak{\frac{1}{2}}^{2}$ & $1/2$& $1/4$ & $1/2$ & $0$ & $0$\\
    \hline
    $(ii)$ & 3 & $\pr{X=k}={^3 C_k}\brak{\frac{1}{2}}^{3}$ & $1/8$ & $3/8$ & $3/8$ & $1/8$ & $0$\\
    \hline
    $(iii)$ & 4 & $\pr{X=k}={^4 C_k}\brak{\frac{1}{2}}^{4}$ & $1/16$& $1/4$ & $3/8$ & $1/4$ & $1/16$\\
    \hline
  \end{tabular}
  \label{3.6:Table1}
}
\end{table}

\item Let X represent the difference between the number of heads and the number of tails obtained when a coin is tossed 6 times. What are possible values of X?\\
\item Six balls are drawn successively from an urn containing 7 red and 9 black balls. Tell whether or not the trials of drawing balls are Bernoulli trials when after each draw the ball drawn is\\
(i) replaced \\
(ii) not replaced in the urn.\\
\solution
Properties to be satisfied if a trial needs to be a bernoulli trial:\\
\begin{enumerate}
    \item Number of trials should be finite.
    \item each trial should have utcomes of success and failure.\
    \item if P is the success probability then failure probability should be 1-P
    \item probability of success should not vary with trial
\end{enumerate}

\textbf{Case(i):Replaced} \\ 
Number of red balls = 7\\
Number of black balls = 9\\
let X be the random variable and 
\begin{itemize}
    \item X=1 is success which is Drawing red ball
    \item X=0 is Failure which is Drawing black ball
\end{itemize}
Success Probability 
\begin{equation}
    P(X=1) = \frac{7}{16} 
\end{equation}
Failure Probability
\begin{equation}
     P(X=0) = \frac{9}{16} = 1-P(X=1)
\end{equation}
 Success Probability is  constant for all Trials.
 as X satisfies all properties of Bernoulli therefore Trials are Bernoulli Trials.\\ \\
\textbf{Case(ii):Not Replaced} \\ \\
In this case Success Probability is 
\begin{equation}
    P(X=1) =\frac{7}{16}
\end{equation}
 for Second Trial 
\begin{equation}
     P(X=1) = \frac{6}{15} 
\end{equation} 
 Corresponding Failure Probabilities are 
 \begin{equation}
     P(X=0) = \frac{9}{16}
 \end{equation} and for 2nd trial 
 \begin{equation}
     P(X=0) = \frac{8}{15}
 \end{equation}  
  probability of success and corresponding failure is varying with trials therefore these are not Bernoulli Trials. 
\begin{itemize}
    \item Case(i):Trials are Bernoulli Trials 
    \item Case(ii): Trials are not Bernoulli Trials
\end{itemize}

\item If a fair coin is tossed 10 times, find the probability of
\begin{enumerate}
\item  exactly six heads
\item  at least six heads
\item  at most six  heads
\end{enumerate}
\solution
%
\begin{align}
    E(X) &= \frac{1}{\sqrt{2\pi}} \int_{-\infty}^{\infty} x e^{-\frac{x^2}{2}}dx\\
    &=0 \quad \brak{ \text{ odd function}}
\end{align}
\begin{align}
    E\brak{X^2}&= \frac{1}{\sqrt{2\pi}}\int_{-\infty}^{\infty} x^2
e^ {-\frac{x^2}{2}} dx \quad \brak{even function}\\
    &= \frac{2}{\sqrt{2\pi}} \int_{0}^{\infty} x^2 e^{-\frac{x^2}{2}} dx\\
    &= \frac{2}{\sqrt{2\pi}}\int_{0}^{\infty}\sqrt{2u}e^{-u} du \quad\brak{Let \frac{x^2}{2}= u}\\
    &= \frac{2}{\sqrt{\pi}} \int_{0}^{\infty} e^{-u} u^{\frac{3}{2}-1} du\\
    &= \frac{2}{\sqrt{\pi}} \Gamma\brak{{\frac{3}{2}}}\\
    &= \frac{1}{\sqrt{\pi}}\Gamma\brak{\frac{1}{2}} \\
    &= 1
\end{align}
where we have used the fact that
\begin{align}
\quad\because \Gamma(n)= (n-1)\Gamma(n-1); \Gamma\brak{\frac{1}{2}}=\sqrt{\pi}
\end{align}
%
Thus, the  variance is
\begin{align}
    \sigma^2 =  E\brak X^2 - E^2\brak X = 1
\end{align}


\item Ten eggs are drawn successively with replacement from a lot containing 10$\%$ defective eggs. Find the probability that there is at least one defective egg.\\
\solution
%
\begin{align}
    E(X) &= \frac{1}{\sqrt{2\pi}} \int_{-\infty}^{\infty} x e^{-\frac{x^2}{2}}dx\\
    &=0 \quad \brak{ \text{ odd function}}
\end{align}
\begin{align}
    E\brak{X^2}&= \frac{1}{\sqrt{2\pi}}\int_{-\infty}^{\infty} x^2
e^ {-\frac{x^2}{2}} dx \quad \brak{even function}\\
    &= \frac{2}{\sqrt{2\pi}} \int_{0}^{\infty} x^2 e^{-\frac{x^2}{2}} dx\\
    &= \frac{2}{\sqrt{2\pi}}\int_{0}^{\infty}\sqrt{2u}e^{-u} du \quad\brak{Let \frac{x^2}{2}= u}\\
    &= \frac{2}{\sqrt{\pi}} \int_{0}^{\infty} e^{-u} u^{\frac{3}{2}-1} du\\
    &= \frac{2}{\sqrt{\pi}} \Gamma\brak{{\frac{3}{2}}}\\
    &= \frac{1}{\sqrt{\pi}}\Gamma\brak{\frac{1}{2}} \\
    &= 1
\end{align}
where we have used the fact that
\begin{align}
\quad\because \Gamma(n)= (n-1)\Gamma(n-1); \Gamma\brak{\frac{1}{2}}=\sqrt{\pi}
\end{align}
%
Thus, the  variance is
\begin{align}
    \sigma^2 =  E\brak X^2 - E^2\brak X = 1
\end{align}


\item Find the mean of the Binomial distribution B(4,$\frac{1}{3}$).
\\
\solution
%
\begin{align}
    E(X) &= \frac{1}{\sqrt{2\pi}} \int_{-\infty}^{\infty} x e^{-\frac{x^2}{2}}dx\\
    &=0 \quad \brak{ \text{ odd function}}
\end{align}
\begin{align}
    E\brak{X^2}&= \frac{1}{\sqrt{2\pi}}\int_{-\infty}^{\infty} x^2
e^ {-\frac{x^2}{2}} dx \quad \brak{even function}\\
    &= \frac{2}{\sqrt{2\pi}} \int_{0}^{\infty} x^2 e^{-\frac{x^2}{2}} dx\\
    &= \frac{2}{\sqrt{2\pi}}\int_{0}^{\infty}\sqrt{2u}e^{-u} du \quad\brak{Let \frac{x^2}{2}= u}\\
    &= \frac{2}{\sqrt{\pi}} \int_{0}^{\infty} e^{-u} u^{\frac{3}{2}-1} du\\
    &= \frac{2}{\sqrt{\pi}} \Gamma\brak{{\frac{3}{2}}}\\
    &= \frac{1}{\sqrt{\pi}}\Gamma\brak{\frac{1}{2}} \\
    &= 1
\end{align}
where we have used the fact that
\begin{align}
\quad\because \Gamma(n)= (n-1)\Gamma(n-1); \Gamma\brak{\frac{1}{2}}=\sqrt{\pi}
\end{align}
%
Thus, the  variance is
\begin{align}
    \sigma^2 =  E\brak X^2 - E^2\brak X = 1
\end{align}


\item The probability of a shooter hitting a target is $\frac{3}{4}$. How many minimum
number of times must he/she fire so that the probability of hitting the target at least
once is more than 0.99?
\\
\solution
%
\begin{align}
    E(X) &= \frac{1}{\sqrt{2\pi}} \int_{-\infty}^{\infty} x e^{-\frac{x^2}{2}}dx\\
    &=0 \quad \brak{ \text{ odd function}}
\end{align}
\begin{align}
    E\brak{X^2}&= \frac{1}{\sqrt{2\pi}}\int_{-\infty}^{\infty} x^2
e^ {-\frac{x^2}{2}} dx \quad \brak{even function}\\
    &= \frac{2}{\sqrt{2\pi}} \int_{0}^{\infty} x^2 e^{-\frac{x^2}{2}} dx\\
    &= \frac{2}{\sqrt{2\pi}}\int_{0}^{\infty}\sqrt{2u}e^{-u} du \quad\brak{Let \frac{x^2}{2}= u}\\
    &= \frac{2}{\sqrt{\pi}} \int_{0}^{\infty} e^{-u} u^{\frac{3}{2}-1} du\\
    &= \frac{2}{\sqrt{\pi}} \Gamma\brak{{\frac{3}{2}}}\\
    &= \frac{1}{\sqrt{\pi}}\Gamma\brak{\frac{1}{2}} \\
    &= 1
\end{align}
where we have used the fact that
\begin{align}
\quad\because \Gamma(n)= (n-1)\Gamma(n-1); \Gamma\brak{\frac{1}{2}}=\sqrt{\pi}
\end{align}
%
Thus, the  variance is
\begin{align}
    \sigma^2 =  E\brak X^2 - E^2\brak X = 1
\end{align}


\item Three coins are tossed simultaneously. Consider the event E "three heads or three tails", F "at least two heads" and G "at most two heads". Of the pairs (E,F), (E,G) and (F,G), which are independent? which are dependent?\\
\solution
\input{./solutions/20-30/chapters/prob/examples/docq22.tex}
\item A die is tossed thrice. Find the probability of getting an odd number at least once.\\
\\
\solution
%
\begin{align}
    E(X) &= \frac{1}{\sqrt{2\pi}} \int_{-\infty}^{\infty} x e^{-\frac{x^2}{2}}dx\\
    &=0 \quad \brak{ \text{ odd function}}
\end{align}
\begin{align}
    E\brak{X^2}&= \frac{1}{\sqrt{2\pi}}\int_{-\infty}^{\infty} x^2
e^ {-\frac{x^2}{2}} dx \quad \brak{even function}\\
    &= \frac{2}{\sqrt{2\pi}} \int_{0}^{\infty} x^2 e^{-\frac{x^2}{2}} dx\\
    &= \frac{2}{\sqrt{2\pi}}\int_{0}^{\infty}\sqrt{2u}e^{-u} du \quad\brak{Let \frac{x^2}{2}= u}\\
    &= \frac{2}{\sqrt{\pi}} \int_{0}^{\infty} e^{-u} u^{\frac{3}{2}-1} du\\
    &= \frac{2}{\sqrt{\pi}} \Gamma\brak{{\frac{3}{2}}}\\
    &= \frac{1}{\sqrt{\pi}}\Gamma\brak{\frac{1}{2}} \\
    &= 1
\end{align}
where we have used the fact that
\begin{align}
\quad\because \Gamma(n)= (n-1)\Gamma(n-1); \Gamma\brak{\frac{1}{2}}=\sqrt{\pi}
\end{align}
%
Thus, the  variance is
\begin{align}
    \sigma^2 =  E\brak X^2 - E^2\brak X = 1
\end{align}

\item A game consists of tossing a one rupee coin 3 times and noting its outcome each time. Hanif wins if all the tosses give the same result i.e., three heads or three tails, and loses otherwise. Calculate the probability that Hanif will lose the game.
\\
\solution
%
\begin{align}
    E(X) &= \frac{1}{\sqrt{2\pi}} \int_{-\infty}^{\infty} x e^{-\frac{x^2}{2}}dx\\
    &=0 \quad \brak{ \text{ odd function}}
\end{align}
\begin{align}
    E\brak{X^2}&= \frac{1}{\sqrt{2\pi}}\int_{-\infty}^{\infty} x^2
e^ {-\frac{x^2}{2}} dx \quad \brak{even function}\\
    &= \frac{2}{\sqrt{2\pi}} \int_{0}^{\infty} x^2 e^{-\frac{x^2}{2}} dx\\
    &= \frac{2}{\sqrt{2\pi}}\int_{0}^{\infty}\sqrt{2u}e^{-u} du \quad\brak{Let \frac{x^2}{2}= u}\\
    &= \frac{2}{\sqrt{\pi}} \int_{0}^{\infty} e^{-u} u^{\frac{3}{2}-1} du\\
    &= \frac{2}{\sqrt{\pi}} \Gamma\brak{{\frac{3}{2}}}\\
    &= \frac{1}{\sqrt{\pi}}\Gamma\brak{\frac{1}{2}} \\
    &= 1
\end{align}
where we have used the fact that
\begin{align}
\quad\because \Gamma(n)= (n-1)\Gamma(n-1); \Gamma\brak{\frac{1}{2}}=\sqrt{\pi}
\end{align}
%
Thus, the  variance is
\begin{align}
    \sigma^2 =  E\brak X^2 - E^2\brak X = 1
\end{align}

\item A die is thrown twice. What is the probability that\\
\begin{enumerate}[label=(\roman*)]
\item  5 will not come up either time? \\
\item  5 will come up at least once?\\
\end{enumerate}
Hint : Throwing a die twice and throwing two dice simultaneously are treated as the
same experiment
\\
\solution
%
\begin{align}
    E(X) &= \frac{1}{\sqrt{2\pi}} \int_{-\infty}^{\infty} x e^{-\frac{x^2}{2}}dx\\
    &=0 \quad \brak{ \text{ odd function}}
\end{align}
\begin{align}
    E\brak{X^2}&= \frac{1}{\sqrt{2\pi}}\int_{-\infty}^{\infty} x^2
e^ {-\frac{x^2}{2}} dx \quad \brak{even function}\\
    &= \frac{2}{\sqrt{2\pi}} \int_{0}^{\infty} x^2 e^{-\frac{x^2}{2}} dx\\
    &= \frac{2}{\sqrt{2\pi}}\int_{0}^{\infty}\sqrt{2u}e^{-u} du \quad\brak{Let \frac{x^2}{2}= u}\\
    &= \frac{2}{\sqrt{\pi}} \int_{0}^{\infty} e^{-u} u^{\frac{3}{2}-1} du\\
    &= \frac{2}{\sqrt{\pi}} \Gamma\brak{{\frac{3}{2}}}\\
    &= \frac{1}{\sqrt{\pi}}\Gamma\brak{\frac{1}{2}} \\
    &= 1
\end{align}
where we have used the fact that
\begin{align}
\quad\because \Gamma(n)= (n-1)\Gamma(n-1); \Gamma\brak{\frac{1}{2}}=\sqrt{\pi}
\end{align}
%
Thus, the  variance is
\begin{align}
    \sigma^2 =  E\brak X^2 - E^2\brak X = 1
\end{align}

\item A die is thrown again and again until three sixes are obtained. Find the probability of obtaining the third six in the sixth throw of the die.\\
\solution
For the 3rd six to occur in the 6th trial, 2 sixes should compulsorily occur in 5 trials.  Defining $X \sim B\brak{5,\frac{1}{6}}$, this probability is given by
\begin{align}
\pr{X = 2} = \comb{5}{2}\brak{\frac{1}{6}}^2\brak{\frac{5}{6}}^3
\end{align}
%
The desired probability is then obtained as
\begin{align}
    \pr{X=2}\times\frac{1}{6} = \frac{625}{23328} 
\end{align}

%
\item Find the probability of throwing at most 2 sixes in 6 throws of a single die.\\
\solution
Defining $X \sim B\brak{6,\frac{1}{5}}$, the desired probability is given by

\begin{align}
\pr{X \leq 2} &=  \sum_{k=0}^{2} \comb{6}{k}\brak{\frac{1}{6}}^k\brak{\frac{5}{6}}^{6-k} \\
&= \brak{\frac{5}{6}}^4 \times \frac{35}{18}\\
&= 0.9377
\end{align}
%
\item In a box containing 100 bulbs, 10 are defective. The probability that out of a
sample of 5 bulbs, none is defective is
\begin{enumerate}
\item $10^{-1}$
\item $\brak{\frac{1}{2}}^5$
\item $\brak{\frac{9}{10}}^5$
\item $\frac{9}{10}$
\end{enumerate}
%
\solution
From the given information, the probability of a bulb being defective is
\begin{align}
p = \frac{10}{100} = \frac{1}{10}
\end{align}
%
Defining $X \sim B \brak{5,\frac{1}{10}}$  the desired probability is
\begin{align}
    \pr{X=0} &= \comb{5}{0} \brak{\frac{1}{10}}^{0} \brak{\frac{9}{10}} ^ {5-0}\\
    &= \brak{\frac{9}{10}}^5
\end{align}

\item In a hurdle race, a player has to cross 10 hurdles. The probability that he will
clear each hurdle is $\frac{5}{6}$. What is the probability that he will knock down fewer than 2 hurdles?\\
\solution 
%
\begin{align}
    E(X) &= \frac{1}{\sqrt{2\pi}} \int_{-\infty}^{\infty} x e^{-\frac{x^2}{2}}dx\\
    &=0 \quad \brak{ \text{ odd function}}
\end{align}
\begin{align}
    E\brak{X^2}&= \frac{1}{\sqrt{2\pi}}\int_{-\infty}^{\infty} x^2
e^ {-\frac{x^2}{2}} dx \quad \brak{even function}\\
    &= \frac{2}{\sqrt{2\pi}} \int_{0}^{\infty} x^2 e^{-\frac{x^2}{2}} dx\\
    &= \frac{2}{\sqrt{2\pi}}\int_{0}^{\infty}\sqrt{2u}e^{-u} du \quad\brak{Let \frac{x^2}{2}= u}\\
    &= \frac{2}{\sqrt{\pi}} \int_{0}^{\infty} e^{-u} u^{\frac{3}{2}-1} du\\
    &= \frac{2}{\sqrt{\pi}} \Gamma\brak{{\frac{3}{2}}}\\
    &= \frac{1}{\sqrt{\pi}}\Gamma\brak{\frac{1}{2}} \\
    &= 1
\end{align}
where we have used the fact that
\begin{align}
\quad\because \Gamma(n)= (n-1)\Gamma(n-1); \Gamma\brak{\frac{1}{2}}=\sqrt{\pi}
\end{align}
%
Thus, the  variance is
\begin{align}
    \sigma^2 =  E\brak X^2 - E^2\brak X = 1
\end{align}

\item Suppose that 90$\%$ of people are right-handed. What is the probability that
at most 6 of a random sample of 10 people are right-handed?\\
\solution
From the given information, the random variable in question is $X \sim B\brak{10,\frac{9}{10}}$.
The desired probability is then given by 
\begin{align}
\pr{X\le 6} & = 1-P(X>6) \\ 
              &  =1-\sum_{k=7}^{10}\comb{10}{k}\brak{\frac{9}{10}}^k\brak{\frac{1}{10}}^{10-k} \\
             &=1-\frac{9^7\times 2064}{10^{10}}\\ 
             &=.0128
\end{align}            
\item The probability that a student is not a swimmer is $\frac{1}{5}$. Then the probability that out of five students, four are swimmers is\\
\begin{enumerate}
\item $\comb{5}{4}\brak{\frac{4}{5}}^4 \frac{1}{5}$
\item $\brak{\frac{4}{5}}^4 \frac{1}{5}$
\item $\comb{5}{1}\brak{\frac{4}{5}}^4 \frac{1}{5}$
\item None of these
\end{enumerate}
\solution
Let X be the number of swimmers and the probability that student is swimmer is
\begin{align}
p=\frac{4}{5}
\end{align}
%
Then the desired probability is
\begin{align}
\pr{X=4}=\comb{5}{4}\brak{\frac{4}{5}}^4\brak{\frac{1}{5}}
\end{align}

%
\item A die is thrown 6 times. If ‘getting an odd number’ is a success, what is the probability of\\
(i) 5 successes?\\
(ii) at least 5 successes?\\
(iii) at most 5 successes?\\
%
\solution
let $X \sim B\brak{5,\frac{1}{2}}$ denote the random variable in question.  Then
\begin{enumerate}
\item 
\begin{align}
\pr{X=5} &= \comb{6}{5}\brak{\frac{1}{2}}^5\brak{\frac{1}{2}}^{6-5}
\\
&=\frac{3}{32}
\end{align}
\item 
\begin{align}
\pr{X \ge 5} &= \sum_{k=5}^{6}\comb{6}{5}\brak{\frac{1}{2}}^k\brak{\frac{1}{2}}^{6-k}
\\
&= 0.109375
\end{align}
\item 
\begin{align}
\pr{X \le 5} &= 1 - \pr{X = 6}
\\
&=1-\comb{6}{6}\brak{\frac{1}{2}}^6\brak{\frac{1}{2}}^{6-6}
\\
&= \frac{63}{64}
\end{align}

\end{enumerate}


\item A bag consists of 10 balls each marked with one of the digits 0 to 9. If four balls are drawn successively with replacement from the bag, what is the probability that none is marked with the digit 0?\\
\solution
Let $X$ be number marked on ball drawn.
Since the balls are drawn with replacement, the trials are Bernoulli trials.
\\
So $X$ has Binomial Distribution 
\begin{equation}
    \pr {X=k}=\comb{n}{k} \times q^{n-k} \times p^{k} 
\end{equation} 

Here,
\begin{align}
& n = \text {number of times we pick the ball}  \\
& p= \text{Probability of getting ball marked as 0} \\
& q=1-p
\end{align}

\begin{table}[h!]
\centering

\begin{tabular}{|c|c|c|c|c|}
\hline
\textbf{Variables} & $n$ & $p$    & $q$    & $k$          \\ \hline
\textbf{Values}    & 4 & 1/10 & 9/10 & 0 \\ \hline
\end{tabular}
\caption{Variables and their values}
\label{4.7:table:1}
\end{table}

Now,

\begin{align}
\pr{X=0} &= \comb{4}{0} \times \brak{\frac{9}{10}}^{(4-0)} \times \brak{\frac{1}{10}}^{0} \\
&=\frac{4!}{(4-0)! 0!} \times 1 \times \brak{\frac{9}{10}}^{4} \\
&=\brak { \frac{9}{10} } ^{4} \\
&= 0.6561
\end{align}
%
\item An urn contains 25 balls of which 10 balls bear a mark 'X' and the remaining 15 bear a mark 'Y'. A ball is drawn at random from the urn, its mark is noted down and it is replaced. If 6 balls are drawn in this way, find the probability that\\
(i) all will bear 'X' mark.\\
(ii) not more than 2 will bear 'Y' mark.\\
(iii) at least one ball will bear 'Y' mark.\\
(iv) the number of balls with 'X' mark and 'Y' mark will be equal.\\
\solution
Let X be the number of balls which have 'X' mark on them\\
Using the expression of binomial distribution
\begin{align}
    P(X = r) = {n \choose r} p^r q^{n-r}\\
    P(X \geq k) = \sum_{r = k}^n {n \choose r}p^r q^{n-r}\\
    P(X \leq k) = \sum_{r = 0}^k {n \choose r}p^r q^{n-r}\\
    n = 6,\quad p = 0.4,\quad q = 0.6
\end{align}
\begin{table}[]
\resizebox{\columnwidth}{!}{
    \begin{tabular}{|c|c|c|c|c|}
        \hline
        n & 6 & 6 & 6 & 6\\
        \hline
        Condition & $P(X = 6)$& $P(X \geq 4)$& $P(X \leq 5)$ & $P(X = 3)$ \\
        \hline
        Value & 0.004096 & 0.1792 & 0.995904 & 0.27648\\
        \hline
        Case & $(i)$ & $(ii)$ & $(iii)$ & $(iv)$\\
        \hline
    \end{tabular}
}
\caption{Probabilities of each case }    \label{tab:my_label}
\end{table}


\item It is known that 10$\%$ of certain articles manufactured are defective. What is the probability that in a random sample of 12 such articles, 9 are defective?\\
\solution $X = B \brak{n,p}$ with $n=12$ and $p=\frac{1}{10}$.  Hence, the desired probability is
%
\begin{align}
  \implies\Pr\brak{X=9} &={12 \choose 9}\times\brak{\frac{1}{10}}^9\times\brak{\frac{9}{10}}^{3} \\  &=\frac{16038}{10^{11}}
\end{align}
\item On a multiple choice examination with three possible answers for each of the five questions, what is the probability that a candidate would get four or more correct answers just by guessing ?
\\
\solution
Let $X_i \in (0,1)$ be a random variable where $X_i=1$ represents a successful guess and $X_i = 0$ represents unsuccessful guess on the $i^{th}$ question.\\
\begin{align}
 p=\frac{1}{3}  \notag \\
X= \sum_{i=1}^{n}X_{i} 
\end{align}

where n is the total number of questions. So, X has a binomial distribution.\\
\begin{align}
\pr{X \geq r} = \sum_{k=r}^{n} \binom{n}{r}p^k(1-p)^{n-k} \label{1.0.2} 
\end{align}

Now, in this case n=5 and r=4. From \eqref{1.0.2}\\
\begin{align}
\pr{X=4}=\frac{10}{243}  \notag \\ 
 \pr{X=5}=\frac{1}{243}   \notag \\ 
 \notag
 \end{align}
Hence, required probability= $\frac{11}{243}$ 

\item Five cards are drawn successively with replacement from a well shuffled deck of 52 cards. What is the probability that\\
(i) all the five cards are spades?\\
(ii) only 3 cards are spades?\\
(iii) none is a spade?\\
\solution 
Let $X_{i} \in (0,1)$ be a random variable which denotes whether spade is drawn at the $i^{th}$ draw.\\
   \begin{align}  
      X=\sum_{i=1}^{i=5}X_{i} 
      \end{align}
     where X denotes the number of spades obtained. \\
    \begin{align}
     Since, \pr{x}=\frac{ \text{number of favourable outcome}}{\text{total number of outcomes}} \notag \\
      \pr{x}=\frac{\comb{13}{x}\times \comb{39}{5-x}}{\comb{52}{5}} \label{5.8-1:2.0.2}
      \end{align}
      

\begin{table}[h]
\resizebox{\columnwidth}{!}{
    \begin{tabular}{|l|l|l|l|l|l|l|}
        \hline
        X    & 0 & 1 & 2  & 3  & 4  & 5                     \\ \hline 

P(X) & $\frac{\comb{13}{0}\times\comb{39}{5}}{\comb{52}{5}}$ & $\frac{\comb{13}{1}\times\comb{39}{4}}{\comb{52}{5}}$ & $\frac{\comb{13}{2}\times\comb{39}{3}}{\comb{52}{5}}$ & $\frac{\comb{13}{3}\times\comb{39}{2}}{\comb{52}{5}}$ & $\frac{\comb{13}{4}\times\comb{39}{1}}{\comb{52}{5}}$ & $\frac{\comb{13}{5}\times\comb{39}{0}}{\comb{52}{5}}$ \\ 
\hline

    \end{tabular}
}
\caption{Probabilities of each case }    \label{5.8-1:tab:my_label}
\end{table}

\item Two dice are thrown simultaneously. If X denotes the number of sixes, find the
expectation of X.\\
\solution
Two dice are thrown simultaneously. If $X$ denotes the number of sixes, find the
expectation of $X$
\section*{SOLUTION :}
 When 2 fair dice are thrown simultaneously we know that each die has 6 possible 
 outcomes and outcome of one dice is independent of the outcome of other dice.
 \\$\therefore$ Total possible outcomes are $^{6}C_{1}\,\times\,^{6}C_{1}=36$
 \null \par \null
Let X be a random variable denoting number of sixes in the above case. Then by Binomial
Distribution 
\begin{align}
    \pr{X=k}&=\binom{n}{k}\,p^k\,\brak{1-p}^{n-k} \label{a} \\
    k&=0,\dots,n \label{b}
\end{align}
\begin{align}
\text{Where}\;k &= 0,1,2\\
              n &= 2 \\
              p &= \text{Probability of outcome 6 on a dice} \\
              p &= \dfrac{1}{6}
\end{align}
\null \par \null
From equation \eqref{a} we obtain the following
\newpage
\begin{align}
\pr{X=0} &= \binom{2}{0}\,\brak{\dfrac{1}{6}}^{0}\,\brak{1-\dfrac{1}{6}}^{2} =\dfrac{25}{36} \\
\pr{X=1} &= \binom{2}{1}\,\brak{\dfrac{1}{6}}^{1}\,\brak{1-\dfrac{1}{6}}^{1} =\dfrac{10}{36} \\
\pr{X=2} &= \binom{2}{2}\,\brak{\dfrac{1}{6}}^{2}\,\brak{1-\dfrac{1}{6}}^{0} =\dfrac{1}{36} \\
\end{align}
The probability distribution table is 
\begin{table}[hbt!]
\begin{tabular}{|l|c|c|c|}
\hline
\multicolumn{1}{|c|}{X} & 0 & 1 & 2 \\ \hline
\pr{X=k}                    &$\dfrac{25}{36}$   &$\dfrac{10}{36}$   &$\dfrac{1}{36}$ \\ \hline
\end{tabular}
\end{table}
\begin{align}
\mathbb{E}(X=k) & =\sum_{k=0}^{n} k\pr{k}\\
    & =\sum_{k=0}^{n} k\,\binom{n}{k}\,p^k\,\brak{1-p}^{n-k}\\ 
    & = n\cdot p\,\sum_{k=1}^{n-1} \binom{n-1}{k-1}\,p^{k-1}\,\brak{1-p}^{(n-1)-(k-1)}\\
    & = n\cdot p\,\brak{1 + (1-p)}^{n-1}\\
    & = n \cdot p \\
 \mathbb{E}(X=k) &= n\cdot p = 2\times\dfrac{1}{6} = \dfrac{1}{3}
\end{align}

\\
\item Find the variance of the number obtained on a throw of an unbiased die.\\
Let $X \in \{1,2,3,4,5,6\}$, be the random variable representing outcome of the die.The probability mass function(pmf) can be expressed as
\begin{align}
p_X\brak{n} = P\brak{X=n} =  \begin{cases}
			\frac{1}{6}, & \text{if $1 \leq n\leq 6$}\\
            0, & \text{otherwise}
		 \end{cases} 
\end{align}



		              
The variance (Var(X)) of this distribution can be found by definition,\\
\begin{align}
Var\brak{X} = E\brak{X^{2}}-\brak{E\brak{X}}^{2} \label{Eq:5.30:1}
\end{align}
where,
\begin{align}
E\brak{X}=\sum_{k=1}^{k=6} kp_X\brak{k}  \\
E\brak{X}=\frac{1}{6}\sum_{k=1}^{k=6} k \label{Eq:5.30:2}
\end{align}
We know that, sum of natural numbers from 1 to n is,
\begin{align}
\sum_{k=1}^{k=n} k = \frac{n\brak{n+1}}{2} \label{Eq:5.30:3}
\end{align}
By substituting the formula from \eqref{Eq:5.30:3} in \eqref{Eq:5.30:2} and n=6, We get,
\begin{align}
E\brak{X}=\frac{1}{6} \times \frac{6\times7}{2}  \\
E\brak{X}=\frac{7}{2} \label{Eq:5.30:4}
\end{align}
And,
\begin{align}
E\brak{X^{2}}=\sum_{k=1}^{k=6} k^{2}p_X\brak{k}  \\
E\brak{X^{2}}=\frac{1}{6}\sum_{k=1}^{k=6} k^{2} \label{Eq:5.30:5}
\end{align}
We know that, sum of squares of natural numbers from 1 to n is,
\begin{align}
\sum_{k=1}^{k=n} k^{2} = \frac{n\brak{n+1}\brak{2n+1}}{6} \label{Eq:5.30:6}
\end{align}
By substituting the formula from \eqref{Eq:5.30:6} in \eqref{Eq:5.30:5} and n=6, We get,
\begin{align}
E\brak{X^{2}}=\frac{1}{6} \times \frac{6\times7\times13}{6}  \\
E\brak{X^{2}}=\frac{91}{6} \label{Eq:5.30:7}
\end{align}
By substituting the values from \eqref{Eq:5.30:7} and \eqref{Eq:5.30:4} in \eqref{Eq:5.30:1}
\begin{align}
Var\brak{X} = E\brak{X^{2}}-\brak{E(X)}^{2}  \\
Var\brak{X} = \frac{91}{6} - \frac{49}{4}  \\
Var\brak{X} = \frac{70}{12}  \\
Var\brak{X} = 2.9167 \label{Eq:5.30:8}
\end{align}
\item A bag contains 2 white and 1 red balls. One ball is drawn at random and then put back in the box after noting its colour. The process is repeated again. If X denotes the number of red balls recorded in the two draws, describe X.\\
Given, a bag containing 2 white and 1 red balls. Let the random variable $X_{i}\in\{0,1\},i=1,2,$ represent the outcome of the colour of the ball drawn in the first, second attempts. $X_{i}=0,X_{i}=1$ denote a white ball, red ball being drawn respectively, in the $i^{th}$ attempt.
\begin{comment}
As the ball drawn in the first attempt is replaced in the bag, for both the attempts, the number of balls of a specified colour, and their probability  mass function's (pmf's) remain the same. i.e, 
\begin{align}
    \tag{5.25.1}
    n(X_{i}=0)=2\\
    \tag{5.25.2}
    n(X_{i}=1)=1\\
    \tag{5.25.3}
    \therefore n(X_{i}=0)+n(X_{i}=1)=3 
\end{align}
and
\begin{align}
    \tag{5.25.4}
    \Pr(X_{i}=j) = 
	\begin{cases}
	\dfrac{2}{3}, &j=0 \\~\\[-1em]
	\dfrac{1}{3}, &j=1 \\~\\[-1em]
	0, & otherwise
	\end{cases}
\end{align}
\newpage
\end{comment}
\newline
\newline
Define 
\begin{align}
    \tag{5.25.1}
    X=X_{1}+X_{2}
\end{align}
so that $X\in\{0,1,2\}$ represents a random variable denoting the number of red balls drawn in both the attempts. Then, $X$ has a binomial distribution with 
\begin{align}
    \tag{5.25.2}
    Pr(X=k)={\comb{n}{k}}p^{k}q^{n-k}
    \label{5.25:eq:binomialdistr}
\end{align}
where,
\begin{align}
    \tag{5.25.3}
    n=2
\end{align}
$p =$ probability of success = probability of drawing a red ball = $Pr(X_{i}=1)$
\begin{align}
    \tag{5.25.4}
    p=\frac{1}{3}
\end{align}
$q =$ probability of failure = $1-p$
\begin{align}
    \tag{5.25.5}
    q=1-p=1-\frac{1}{3}=\frac{2}{3}
\end{align}
\newline
Hence, on substituting and simplifying, we get
\begin{align}
    \tag{5.25.6}
    Pr(X=0)=\frac{4}{9}\\
    \tag{5.25.7}
    Pr(X=1)=\frac{4}{9}\\
    \tag{5.25.8}
    Pr(X=2)=\frac{1}{9}
\end{align}
\newline
Using \eqref{5.25:eq:binomialdistr}, we get the following probability distribution.
\begin{table}[h!]
\centering
\caption{Probability distribution of X}
\label{5.25:table:1}
\begin{tabular}{|c||c|c|c|}
    \hline
    Condition & $X = 0$& $X =1 $& $X=2$ \\
    \hline
    & & &\\
    Probability & $\comb{2}{0}p^{0}q^{2}$ & $\comb{2}{1}p^{1}q^{1}$ & $\comb{2}{2}p^{2}q^{0}$\\[1ex]
    \hline
\end{tabular}
\end{table}
\item Find the probability of throwing at most 2 sixes in 6 throws of a single die.\\
\solution
Let X represent the number of sixes in six throws of a dice
\newline
      X $\in$ \{0,1,2,3,4,5,6\} \\
By Binomial distribution formula, \newline
\begin{align}
P(X=k) = \comb{n}{k}  p^{k} (1-p)^{n-k}
\end{align}
Here,
\begin{table}[!ht]
\centering
\begin{tabular}{|c|c|}
\hline
Symbol & Meaning  \\ \hline
k                      &    no. of sixes in six throws of a dice                 \\ \hline
n                      & no. of throws = 6                    \\ \hline
p                      &  Pr of getting 6 in single throw=\( \frac{1}{6} \)                  \\ \hline
\end{tabular}

\caption{This table gives the meaning of each symbol used in the formula}
\label{tab:Table 5.10}
\end{table}
\begin{align}
\pr{X\leq 2}&=\pr{X=0}+\pr{X=1}+\pr{X=2}\\
 \pr{X=0}&=\comb{6}{0} \times \brak{\frac{1}{6}}^{0} \times  \brak{\frac{5}{6}} ^{6-0}\\
\pr{X=1}&=\comb{6}{1} \times \brak{\frac{1}{6}}^{1} \times  \brak{\frac{5}{6}} ^{6-1}\\
 \pr{X=2}&=\comb{6}{2} \times \brak{\frac{1}{6}}^{2} \times  \brak{\frac{5}{6}} ^{6-2}\\
\pr{X\leq 2} &= \brak{ \frac{5^6}{6^6} } \times 1 + \brak{ \frac{5^5}{6^6} } \times 6 +  \brak{\frac{5^4}{6^6} } \times 15 \\
&= 0.937714 \nonumber\\
\end{align}
\end{enumerate}


 
\section{Uniform Distribution }
%\renewcommand{\thefigure}{\theenumi}
%\renewcommand{\thetable}{\theenumi}
\renewcommand{\theequation}{\theenumi}
\begin{enumerate}[label=\thesection.\arabic*.,ref=\thesection.\theenumi]
\numberwithin{equation}{enumi}
\numberwithin{table}{enumi}


\item Two dice, one blue and one grey, are thrown at the same time.  
\begin{enumerate}
\item  Complete Table \ref{table:1.2.133}.
\item  A student argues that there are 11 possible outcomes 2, 3, 4, 5, 6, 7, 8, 9, 10, 11 and 12. Therefore, each of them has a probability $\frac{1}{11}$. Do you agree with this argument? Justify your answer.
\end{enumerate}
%
\begin{table}[ht!]
\centering
\input{./solutions/10-1/prob/codes/tables/input.tex}
\caption{Input Values}
\label{table:1.2.133}	
\end{table}
\item A die is thrown once. Find the probability of getting\\
(i) a prime number;\\
(ii) a number lying between 2 and 6;\\
(iii) an odd number.
\\
\solution
%
\begin{align}
    E(X) &= \frac{1}{\sqrt{2\pi}} \int_{-\infty}^{\infty} x e^{-\frac{x^2}{2}}dx\\
    &=0 \quad \brak{ \text{ odd function}}
\end{align}
\begin{align}
    E\brak{X^2}&= \frac{1}{\sqrt{2\pi}}\int_{-\infty}^{\infty} x^2
e^ {-\frac{x^2}{2}} dx \quad \brak{even function}\\
    &= \frac{2}{\sqrt{2\pi}} \int_{0}^{\infty} x^2 e^{-\frac{x^2}{2}} dx\\
    &= \frac{2}{\sqrt{2\pi}}\int_{0}^{\infty}\sqrt{2u}e^{-u} du \quad\brak{Let \frac{x^2}{2}= u}\\
    &= \frac{2}{\sqrt{\pi}} \int_{0}^{\infty} e^{-u} u^{\frac{3}{2}-1} du\\
    &= \frac{2}{\sqrt{\pi}} \Gamma\brak{{\frac{3}{2}}}\\
    &= \frac{1}{\sqrt{\pi}}\Gamma\brak{\frac{1}{2}} \\
    &= 1
\end{align}
where we have used the fact that
\begin{align}
\quad\because \Gamma(n)= (n-1)\Gamma(n-1); \Gamma\brak{\frac{1}{2}}=\sqrt{\pi}
\end{align}
%
Thus, the  variance is
\begin{align}
    \sigma^2 =  E\brak X^2 - E^2\brak X = 1
\end{align}

\item In a game, a man wins a rupee for a six and loses a rupee for any other number when a fair die is thrown. The man decided to throw a die thrice but to quit as and when he gets a six. Find the expected value of the amount he wins / loses.\\
\item 
\item Find the probability of getting 5 exactly twice in 7 throws of a die.\\
\solution
There are 6 outcomes when we throw a die, which are independent of one another. The probability of getting 5 on the die 
\begin{align}
    p = \frac{1}{6}
\end{align}
The die is thrown 7 times and are not dependent on one another \begin{align}
    n = 7
\end{align}
Let the Random Variable be $X$ denote the number of 5s in 7 throws
By Binomial Distribution, we have
\begin{align}
    \Pr(X=k) = \binom{n}{k}p^k(1-p)^{n-k}
\end{align}
We should get 5 exactly twice, so $ k = 2 $\\
\begin{center}
\begin{table}[h]
\caption{Definition of the variables}
\centering
\resizebox{\columnwidth}{!}{%
    \begin{tabular}{|c|c|}
        \hline
         \multicolumn{2}{|c|}{Variables} \\
        \hline
        $p$ & Probability that outcome is 5 when we throw the die\\
        \hline
        $n$ & No of times the die is thrown\\
        \hline
        $X$ & Random Variable denoting the number of 5s out of n number of throws\\
        \hline
        $k$ & Required number of times 5s appear on the die which is 2\\
        \hline
    \end{tabular}
    }
    \label{tab:1}
\end{table}
\end{center}
The probability of getting 5 exactly twice in 7 throws of a die is given by 
\begin{align}
    \Pr(X=2) &= \comb{^7}{C}{_2}\brak{\frac{1}{6}}^2 \brak{\frac{5}{6}}^5 \\[7pt]
    \Pr(X=2) &= 0.234428
\end{align}


\item 

\item
\item 
\item Let X denote the sum of the numbers obtained when two fair dice are rolled. Find the variance and standard deviation of X.\\
\solution
When two fare dice are rolled. The sum of the numbers obtained can have the values 2, 3, 4, 5, 6, 7, 8, 9, 10, 11, 12.\\
$\pr{X}$ = probability of obtaining X as the sum and let us represent the case when first dice shows the number $x_1$ and the second dice shows the number $x_2$ as $(x_1,x_2)$.
\begin{table}[hbt!]
\resizebox{\columnwidth}{!}{
\begin{tabular}{|l|c|c|c|c|c|c|c|c|c|c|c|}
\hline
\multicolumn{1}{|c|}{x} & 2 & 3 & 4 & 5 & 6 & 7 & 8 & 9 & 10 & 11 & 12 \\ \hline
$\pr{X}$                   &$\frac{1}{36}$   &$\frac{2}{36}$   &$\frac{3}{36}$   &$\frac{4}{36}$   &$\frac{5}{36}$   &$\frac{6}{36}$   &$\frac{5}{36}$   &$\frac{4}{36}$   &$\frac{3}{36}$    &$\frac{2}{36}$    &$\frac{1}{36}$    \\ \hline
\end{tabular}
}
\caption{Probability Distribution Table of X}
\label{table:1}
\end{table}
For the above problem,we know that.
\begin{align}
p_x\brak{n} &= 
  \begin{cases}
    0 & \text{if } n \leq 1,\\
    \frac{n-1}{36} & \text{if } 2 \leq n \leq 7,\\
    \frac{13-n}{36} & \text{if } 7 < n \leq 12,\\
    0 & \text{if } n>12.
  \end{cases}
\end{align}
\begin{align}
    &Mean,E(X) \nonumber\\
    & =\sum_{k=1}^{12} (k \times p_x\brak{k})\\
    & = \sum_{k=1}^{6}k\times\frac{1}{36}[k-1] + \sum_{k=7}^{12}k\times\frac{1}{36}[13-k]\\
    & = \frac{1}{36}\left[\sum_{k=1}^{6}k(k-1) + \sum_{k=7-6}^{12-6}(k+6)\times[13-(k+6)]\right]\\
    & = \frac{1}{36}\left[\sum_{k=1}^{6}k(k-1) + \sum_{k=1}^{6}(k+6)\times[13-(k+6)]\right]\\
    &= \frac{1}{36}\sum_{k=1}^{6}\left(k(k-1) + (k+6)(7-k)\right)\\
    &= \frac{1}{36}\sum_{k=1}^{6}\left((k^2- k)+ (7k-k^2+42-6k)\right)\\
    &= \frac{1}{36}\sum_{k=1}^{6}\left( 42\right)\\
    &= \frac{1}{36}\left[ 42 \times 6 \right]
\end{align}
Therefore,
 Mean, $E\brak{X} = 7$
 
\begin{align}
  Variance,\sigma^2 &= E\brak{X-E\brak{X}}^2 \\
    &= E\brak{X^2} - \brak{E\brak{X}}^2
\end{align}
Let us consider $E\brak{X^2}$,
\begin{align}
    & E\brak{X^2} \nonumber\\
    &= \left(\sum_{k=1}^{12}(k^2\times p_x(k))\right)\\
    &= \sum_{k=1}^{6}k^2\times\frac{1}{36}[k-1] + 
    \sum_{k=7}^{12}(k)^2\times[13-k]\\
    &\text{by rearrangement we get}\\
    &= \frac{1}{36}\left[\sum_{k=1}^{6}k^2(k-1) + \sum_{k=7-6}^{12-6}(k+6)^2\times[13-(k+6)]\right]\\
    &= \frac{1}{36}\left[\sum_{k=1}^{6}k^2(k-1) + \sum_{k=1}^{6}(k+6)^2(7-k)\right]\\
    &= \frac{1}{36}\sum_{k=1}^{6}\left(k^2(k-1) + (k^2+36+12k)(7-k)\right)\\
    &= \frac{1}{36}\sum_{k=1}^{6}\left((k^3-k^2) + (-k^3-5k^2+48k+252)\right)\\
    &= \frac{1}{36}\sum_{k=1}^{6}(-6k^2+48k+252)\\
    &= \frac{1}{6}\sum_{k=1}^{6}(-k^2+8k+42)\\
    &= \frac{1}{6}\left[ -\frac{(6)(7)(13)}{6} + 8\times\frac{(6)(7)}{2}+ (42)(6) \right]\\
    &= \frac{1}{6}[-91+168+252]\\
    &= \frac{329}{6}
\end{align}

\begin{align}
    \text{ Variance, }\sigma^2 &=E\brak{X^2} - \brak{E\brak{X}}^2\\
    &= \frac{329}{6}- ((7)^2)\\
    &= \frac{329}{6} - 49\\
   \sigma^2 &= \frac{35}{6}
\end{align}
%
\item Two numbers are selected at random (without replacement) from the first six positive integers. Let X denote the larger of the two numbers obtained. Find E(X).\\
%
\solution
The question can be seen as choosing a number first from 1 to 6 numbers and then choosing one more from the remaining 5 numbers, Let $X_1$ be the $1^{st}$ numbers drawn randomly from 1 to 6 and $X_2$be the $2^{nd}$ number drawn from remaining and $X = \text{max } (X_1,X_2)$
%
\begin{align}
& Pr(X_1=n_1)= \begin{cases}
\dfrac{1}{6},  \text{ if } 1 \leq n_1 \leq 6\\
0,  \text{  otherwise }
\end{cases}\\
& Pr(X_2=n_2)= \begin{cases}
\dfrac{1}{5},  \text{ if } 1 \leq n_2 \leq 6 \text{ and }n_2 \neq n_1\\
0,  \text{  otherwise }
\end{cases}
\end{align}

let max $(X_1,X_2)=i$ and $Pr(X=i)$ denotes the probability that $X = \text{max } (X_1,X_2)=i$
 \begin{multline}
Pr(X=i)=Pr(X_1=i\text{ and }X_2<i)\\
 +Pr(X_2=i\text{ and }X_1<i) \label{eqn_(0.0.1)}
\end{multline}
 

since choosing of $X_1,X_2$ are independent events, so we can write 
$$Pr(X_1 \text{ and }X_2)=Pr(X_1)Pr(X_2)$$
Substituting this in \eqref{eqn_(0.0.1)} gives us
\begin{multline}
Pr(X=i)=Pr(X_1=i)Pr(X_2<i)+\\
Pr(X_2=i)Pr(X_1<i)
\end{multline}

\begin{align}
& \implies Pr(X=i)=\dfrac{1}{6}\times \dfrac{(i-1)}{5}+\dfrac{(i-1)}{6} \times\dfrac{1}{5}\\
& \implies Pr(X=n)=\dfrac{(i-1)}{15}
\end{align}

The expectation value of X represented by E(X) is given by
$$E(X)=\sum_{i=1}^{6} Pr(X=i)\times i$$
\begin{align}
& \implies E(X)=\sum_{i=1}^{6} \dfrac{(i-1)}{15}\times i\\
& \implies E(X)=\sum_{i=1}^{6} \dfrac{(i^2-i)}{15}\\
& \implies E(X)=\dfrac{1}{15} \sum_{i=1}^{6} i^2-\dfrac{1}{15}\sum_{i=1}^{6} i\\
& \implies E(X)=\dfrac{1}{15} \times 91-\dfrac{1}{15} \times 21\\
& \implies E(X)= \textbf{4.6667}
\end{align}

\item
\item Determine P(E/F), if a die is thrown three times,\\
E : 4 appears on the third toss, F : 6 and 5 appears respectively on first two tosses\\

\item In a musical chair game, the person playing the music has been
advised to stop playing the music at any time within 2 minutes after she starts playing.What is the probability that the music will stop within the first half-minute after starting?
\solution
Let the random variable  $X\in\mathbb{R^+} $ represent the time between starting the music and stopping in minutes
For a uniform probability distribution, the Probability Density Function(pdf) is given by
\begin{align}
  p_{X}(x) =
  \begin{cases}
  \dfrac{1}{b-a} = \dfrac{1}{2} & \text{if $0 \leq x \leq 2$}\\ \vspace{-0.1cm}
  0 & \text{otherwise} 
  \end{cases}
\end{align}
The probability that the music will stop within the first half-minute after starting
\begin{align}
    \pr{X=x \,\,|\,\, 0 \leq x \leq 0.5} = \int_0^{\frac{1}{2}} \frac{1}{2} dx = \frac{1}{4} =0.25
\end{align}
\item A missing helicopter is reported to have crashed somewhere in the rectangular region shown in Fig. 15.2. What is the probability that it crashed inside the
lake shown in the figure?
\begin{figure}[!ht]
\centering
\includegraphics[width=\columnwidth]{./prob/figs/lake.eps}
\caption{}
\label{fig:lake}
\end{figure}
\solution
\begin{table}[h!]
    \label{Table-1}
    \caption{Dimensions}
    \centering
   
    \begin{tabular}{|c|c|}
    \hline
        variables & Description\\
        \hline
        l&Length of the lake\\
        \hline
        w&Width of the lake\\
        \hline
        a&Area of the lake\\
        \hline
        L&Length of the whole region\\
        \hline
        W&Width of the whole region\\
        \hline
        A&Area of the whole region\\
        \hline
    \end{tabular}
\end{table}
\begin{align}
l &= (9-6)kms\\
  &= 3kms 
\end{align}
\begin{align}
w&=(4.5-2)kms\\
 &=2.5kms
\end{align}
As the lake is rectangular;
\begin{align}
a &={l}\times{w}\\
&={3kms}\times{2.5kms}\\
&=7.5sq.kms
\end{align}
$$L=9kms$$
$$W=4.5kms$$
The whole region is of rectangular shape. Hence the area of the whole region is;
\begin{align}
A &={L}\times{W}\\
&={9kms}\times{4.5kms}\\
&=40.5sq.kms
\end{align}
\begin{table}[h!]
    \label{Table-2}
    \caption{Events and Probabilities}
    \centering
    \begin{tabular}{|c|c|}
    \hline
        variables & Description\\
        \hline
        X&Helicopter getting crashed inside lake\\
        \hline
        Y&Helicopter getting crashed outside lake\\
        \hline
        P(X)&Probability of occurrence of X\\
        \hline
        P(Y)&Probability of occurrence of Y\\
        \hline
    \end{tabular}
\end{table}
\begin{align}
    P(X)&=\frac{a}{A}\\
    &=\frac{7.5sq.kms}{40.5sq.kms}\\
    &=\frac{5}{27}=0.185
\end{align}

   \item On one page of a telephone directory, there were 200 telephone numbers.
The frequency distribution of their unit place digit (for example, in the number 25828573, the unit place digit is 3) is given in Table \ref{table:prob_exam4}
below

\begin{table}[!ht]
\centering
\resizebox{\columnwidth}{!}{
\begin{tabular}{ |c|c|c|c|c|c|c|c|c|c|c| } 
\hline
 \textbf{Digit} &0 &1 &2 &3 &4 &5 &6 &7 &8 &9 \\ 
 \hline
 \textbf{Frequency} &22 &26 &22 &22 &20 &10 &14 &28 &16 &20 \\ 
 \hline
\end{tabular}
}
\caption{}
\label{table:prob_exam4}
\end{table}
Without looking at the page, the pencil is placed on one of these numbers, i.e., the number is chosen at random. What is the probability that the digit in its unit place is 6?\\
\solution
%
\begin{align}
    E(X) &= \frac{1}{\sqrt{2\pi}} \int_{-\infty}^{\infty} x e^{-\frac{x^2}{2}}dx\\
    &=0 \quad \brak{ \text{ odd function}}
\end{align}
\begin{align}
    E\brak{X^2}&= \frac{1}{\sqrt{2\pi}}\int_{-\infty}^{\infty} x^2
e^ {-\frac{x^2}{2}} dx \quad \brak{even function}\\
    &= \frac{2}{\sqrt{2\pi}} \int_{0}^{\infty} x^2 e^{-\frac{x^2}{2}} dx\\
    &= \frac{2}{\sqrt{2\pi}}\int_{0}^{\infty}\sqrt{2u}e^{-u} du \quad\brak{Let \frac{x^2}{2}= u}\\
    &= \frac{2}{\sqrt{\pi}} \int_{0}^{\infty} e^{-u} u^{\frac{3}{2}-1} du\\
    &= \frac{2}{\sqrt{\pi}} \Gamma\brak{{\frac{3}{2}}}\\
    &= \frac{1}{\sqrt{\pi}}\Gamma\brak{\frac{1}{2}} \\
    &= 1
\end{align}
where we have used the fact that
\begin{align}
\quad\because \Gamma(n)= (n-1)\Gamma(n-1); \Gamma\brak{\frac{1}{2}}=\sqrt{\pi}
\end{align}
%
Thus, the  variance is
\begin{align}
    \sigma^2 =  E\brak X^2 - E^2\brak X = 1
\end{align}


\item Suppose we throw a die once. (i) What is the probability of getting a number greater than 4 ? (ii) What is the probability of getting a number less than or
equal to 4 ?
\\
\solution
%
\begin{align}
    E(X) &= \frac{1}{\sqrt{2\pi}} \int_{-\infty}^{\infty} x e^{-\frac{x^2}{2}}dx\\
    &=0 \quad \brak{ \text{ odd function}}
\end{align}
\begin{align}
    E\brak{X^2}&= \frac{1}{\sqrt{2\pi}}\int_{-\infty}^{\infty} x^2
e^ {-\frac{x^2}{2}} dx \quad \brak{even function}\\
    &= \frac{2}{\sqrt{2\pi}} \int_{0}^{\infty} x^2 e^{-\frac{x^2}{2}} dx\\
    &= \frac{2}{\sqrt{2\pi}}\int_{0}^{\infty}\sqrt{2u}e^{-u} du \quad\brak{Let \frac{x^2}{2}= u}\\
    &= \frac{2}{\sqrt{\pi}} \int_{0}^{\infty} e^{-u} u^{\frac{3}{2}-1} du\\
    &= \frac{2}{\sqrt{\pi}} \Gamma\brak{{\frac{3}{2}}}\\
    &= \frac{1}{\sqrt{\pi}}\Gamma\brak{\frac{1}{2}} \\
    &= 1
\end{align}
where we have used the fact that
\begin{align}
\quad\because \Gamma(n)= (n-1)\Gamma(n-1); \Gamma\brak{\frac{1}{2}}=\sqrt{\pi}
\end{align}
%
Thus, the  variance is
\begin{align}
    \sigma^2 =  E\brak X^2 - E^2\brak X = 1
\end{align}

\item  Given that the two numbers appearing on throwing two dice are different. Find the probability of the event `the sum of numbers on the dice is 4'.\\
\solution
\input{./solutions/20-30/chapters/prob/exercises/docq25.tex}
\item A game of chance consists of spinning an arrow which comes to rest pointing at one of the numbers 1, 2, 3, 4, 5, 6, 7, 8 (see Fig. \ref{fig:122} ), and these are equally likely outcomes. What is the probability that it will point at\\
(i) 8 ?\\
(ii) an odd number?\\
(iii) a number greater than 2?\\
(iv) a number less than 9?\\
\begin{figure}[!ht]
\centering
\includegraphics[width=\columnwidth]{./prob/figs/clock.eps}
\caption{}
\label{fig:122}
\end{figure}
\\
\solution
%
\begin{align}
    E(X) &= \frac{1}{\sqrt{2\pi}} \int_{-\infty}^{\infty} x e^{-\frac{x^2}{2}}dx\\
    &=0 \quad \brak{ \text{ odd function}}
\end{align}
\begin{align}
    E\brak{X^2}&= \frac{1}{\sqrt{2\pi}}\int_{-\infty}^{\infty} x^2
e^ {-\frac{x^2}{2}} dx \quad \brak{even function}\\
    &= \frac{2}{\sqrt{2\pi}} \int_{0}^{\infty} x^2 e^{-\frac{x^2}{2}} dx\\
    &= \frac{2}{\sqrt{2\pi}}\int_{0}^{\infty}\sqrt{2u}e^{-u} du \quad\brak{Let \frac{x^2}{2}= u}\\
    &= \frac{2}{\sqrt{\pi}} \int_{0}^{\infty} e^{-u} u^{\frac{3}{2}-1} du\\
    &= \frac{2}{\sqrt{\pi}} \Gamma\brak{{\frac{3}{2}}}\\
    &= \frac{1}{\sqrt{\pi}}\Gamma\brak{\frac{1}{2}} \\
    &= 1
\end{align}
where we have used the fact that
\begin{align}
\quad\because \Gamma(n)= (n-1)\Gamma(n-1); \Gamma\brak{\frac{1}{2}}=\sqrt{\pi}
\end{align}
%
Thus, the  variance is
\begin{align}
    \sigma^2 =  E\brak X^2 - E^2\brak X = 1
\end{align}


\item Find the variance of the number obtained on a throw of an unbiased die.
\solution
%Let $X \in \{1,2,3,4,5,6\}$, be the random variable representing outcome of the die.The probability mass function(pmf) can be expressed as
\begin{align}
p_X\brak{n} = P\brak{X=n} =  \begin{cases}
			\frac{1}{6}, & \text{if $1 \leq n\leq 6$}\\
            0, & \text{otherwise}
		 \end{cases} 
\end{align}



		              
The variance (Var(X)) of this distribution can be found by definition,\\
\begin{align}
Var\brak{X} = E\brak{X^{2}}-\brak{E\brak{X}}^{2} \label{Eq:5.30:1}
\end{align}
where,
\begin{align}
E\brak{X}=\sum_{k=1}^{k=6} kp_X\brak{k}  \\
E\brak{X}=\frac{1}{6}\sum_{k=1}^{k=6} k \label{Eq:5.30:2}
\end{align}
We know that, sum of natural numbers from 1 to n is,
\begin{align}
\sum_{k=1}^{k=n} k = \frac{n\brak{n+1}}{2} \label{Eq:5.30:3}
\end{align}
By substituting the formula from \eqref{Eq:5.30:3} in \eqref{Eq:5.30:2} and n=6, We get,
\begin{align}
E\brak{X}=\frac{1}{6} \times \frac{6\times7}{2}  \\
E\brak{X}=\frac{7}{2} \label{Eq:5.30:4}
\end{align}
And,
\begin{align}
E\brak{X^{2}}=\sum_{k=1}^{k=6} k^{2}p_X\brak{k}  \\
E\brak{X^{2}}=\frac{1}{6}\sum_{k=1}^{k=6} k^{2} \label{Eq:5.30:5}
\end{align}
We know that, sum of squares of natural numbers from 1 to n is,
\begin{align}
\sum_{k=1}^{k=n} k^{2} = \frac{n\brak{n+1}\brak{2n+1}}{6} \label{Eq:5.30:6}
\end{align}
By substituting the formula from \eqref{Eq:5.30:6} in \eqref{Eq:5.30:5} and n=6, We get,
\begin{align}
E\brak{X^{2}}=\frac{1}{6} \times \frac{6\times7\times13}{6}  \\
E\brak{X^{2}}=\frac{91}{6} \label{Eq:5.30:7}
\end{align}
By substituting the values from \eqref{Eq:5.30:7} and \eqref{Eq:5.30:4} in \eqref{Eq:5.30:1}
\begin{align}
Var\brak{X} = E\brak{X^{2}}-\brak{E(X)}^{2}  \\
Var\brak{X} = \frac{91}{6} - \frac{49}{4}  \\
Var\brak{X} = \frac{70}{12}  \\
Var\brak{X} = 2.9167 \label{Eq:5.30:8}
\end{align}
\end{enumerate}


 
\section{Miscellaneous Distributions}
\begin{enumerate}[label=\thesection.\arabic*.,ref=\thesection.\theenumi]
\numberwithin{equation}{enumi}
\numberwithin{figure}{enumi}

\item A carton consists of 100 shirts of which 88 are good, 8 have minor defects and 4 have major defects.Jimmy, a trader, will only accept the shirts which are good, but Sujatha, another trader, will only reject the shirts which have major defects.One shirt is drawn at random from the carton. What is the probability that\\
(i) it is acceptable to Jimmy?\\
(ii) it is acceptable to Sujatha?
\\
\solution
From the given information, Table \ref{table:5.19} can be generated.
\begin{table}[!ht]
\centering
\resizebox{\columnwidth}{!}{
\begin{tabular}{|p{4cm}|p{4cm}|}
\hline
\multicolumn{2}{|c|}{\textbf{A Random variable which has 3 possible values}}\\ \hline
\multirow{2}{*}{$\pr{X=0}=\frac{4}{100}=0.04$} & out of 100, 4 have major defects shirts
\\
\hline
\multirow{2}{*}{$\pr{X=1}=\frac{8}{100}=0.08$} & out of 100,8 are accepted minor defected shirts\\ \hline
\multirow{2}{*}{$\pr{X=2}=\frac{88}{100}=0.88$} & out of 100,88 are accepted good shirts\\ \hline
\end{tabular}
}
\caption{Random variables}
\label{table:5.19}
\end{table}
Then
\begin{enumerate}
\item The probability that the shirt is acceptable to Jimmy is
\begin{align}
\pr{X=2}=\frac{88}{100}
\end{align}
%
\item The probability that the shirt is acceptable to Sujatha is
\begin{align}
1-\pr{X=0}=1-\frac{4}{100} = \frac{96}{100}
\end{align}

\end{enumerate}
%
The following code simulates the probability
\begin{lstlisting}
codes/misc/discrete.ipynb
\end{lstlisting}
\end{enumerate} 
\section{Axioms of Probability}
\subsection{Boolean Logic}
If A and B are two events such that $\pr{A} = \frac{1}{4}, \pr{B} = \frac{1}{2}$ and $\pr{A B} = \frac{1}{8}$. find $\pr{\text{not A and not B}}$.
\renewcommand{\theequation}{\theenumi}
\begin{enumerate}[label=\thesubsection.\arabic*.,ref=\thesubsection.\theenumi]
\numberwithin{equation}{enumi}

\item 
\begin{align}
A^{\prime}B^{\prime} &=  \brak{A+B}^{\prime}
\\
\implies \pr{A^{\prime}B^{\prime}} &=  \pr{\brak{A+B}^{\prime}} 
\\
&= 1 - \pr{A+B} 
\label{eq:axiom_sum_one}
\end{align}
\item 
\begin{align}
\because A+B &= A\brak{B+B^{\prime}} + B
\\
&= B \brak{A +1} + A B^{\prime}
\\
&=B + A B^{\prime}
\\
\implies \pr{A+B} &= \pr{B + A B^{\prime} }
\\
&=\pr{B}+\pr{ A B^{\prime} } 
\\
&\because B \brak{ A B^{\prime} } = 0
\label{eq:axiom_sum_two}
\end{align}
\item 
\begin{align}
A = A \brak{B+B^{\prime}} =  AB + AB^{\prime}
\label{eq:axiom_sum_A}
\end{align}
and 
\begin{align}
\brak{ AB}\brak{  AB^{\prime}} = 0, \because BB^{\prime} = 0
\label{eq:axiom_sum_AB0}
\end{align}
Hence, $AB$ and $AB^{\prime}$ are mutually exclusive and 
%
\begin{align}
\pr{A} = \pr{AB} + \pr{AB^{\prime}}
\\
\implies 
\pr{AB^{\prime}} =  \pr{A} - \pr{AB}
\label{eq:axiom_sum_ABp}
\end{align}
\item Substituting \eqref{eq:axiom_sum_ABp} in \eqref{eq:axiom_sum_two}, 
\begin{align}
\pr{A+B} &= \pr{A} + \pr{B} - \pr{AB} 
\label{eq:axiom_sum_AB}
\end{align}
\item Substituting \eqref{eq:axiom_sum_AB} in \eqref{eq:axiom_sum_one}
\begin{align}
\pr{A^{\prime}B^{\prime}} &=  1 - \cbrak{\pr{A} + \pr{B} - \pr{AB} }
\\
&= 1 - \brak{\frac{1}{4} + \frac{1}{2} - \frac{1}{8}}
\\
&= \frac{3}{8}
\label{eq:axiom_sum_final}
\end{align}
\end{enumerate}
\subsection{Independent Events}
\renewcommand{\theequation}{\theenumi}
\begin{enumerate}[label=\thesubsection.\arabic*.,ref=\thesubsection.\theenumi]
\numberwithin{equation}{enumi}



\item Prove that if $E$ and $F$ are independent events, then so are the events $E$ and $F^{\prime}$.\\
\solution  If $E$ and $F$ are independent,
\begin{align}
\pr{EF} = \pr{E}\pr{F}
\label{eq:axiom_indep}
\end{align}
From 
\eqref{eq:axiom_sum_AB0}
%
\begin{align}
\pr{EF^{\prime}} =  \pr{E} - \pr{EF}
\label{eq:axiom_indep_EFp}
\end{align}
Substituting from \eqref{eq:axiom_indep} in \eqref{eq:axiom_indep_EFp},
%
\begin{align}
\pr{EF^{\prime}} &=  \pr{E} \brak{1- \pr{F}}
&= \pr{E} \pr{F^{\prime}}
\label{eq:axiom_indep_EFp_ind}
\end{align}
%
\begin{align}
\because FF^{\prime} = 0, F + F^{\prime} = 1
\\
\implies \pr{F}+\pr{F^{\prime}} = 1
\label{eq:axiom_FFp}
\end{align}
By definition, from \eqref{eq:axiom_indep_EFp_ind}, we conclude that $E$ and $F^{\prime}$ are independent.
\item If A and B are two independent events, then the probability of occurrence of at least one of A and B is given by 1- $P(A^{\prime}) P(B^{\prime})$\\
\solution 
\begin{align}
\because (A+B)(A+B)^{\prime} = 0
\\
\implies 1 = \pr{A+B} + \pr{\brak{A+B}^{\prime}}
\\
\implies \pr{A+B} = 1 - \pr{A^{\prime}B^{\prime}} 
\\
= 1 - \pr{A^{\prime}}\pr{B^{\prime}} 
\end{align}
using the definition of independence.
\end{enumerate}
\subsection{Conditional Probability}
\renewcommand{\theequation}{\theenumi}
\begin{enumerate}[label=\thesubsection.\arabic*.,ref=\thesubsection.\theenumi]
\numberwithin{equation}{enumi}

\item Given that E and F are events such that P(E) = 0.6, P(F) = 0.3 and P(E  F) = 0.2, find $\pr{E|F}$ and $\pr{F|E}$\\
\solution By definition,
\begin{align}
\pr{E|F} = \frac{\pr{EF}}{\pr{F}} = \frac{0.2}{0.3} = \frac{2}{3}
\end{align}
%
Similarly,
\begin{align}
\pr{F|E} = \frac{\pr{EF}}{\pr{E}} = \frac{1}{3}
\end{align}


\item A fair die is rolled. Consider the events E =  (1, 3, 5), F = (2, 3) and G = (2, 3, 4, 5) Find\\
\begin{enumerate}
\item  $\pr{E|F}$ and $\pr{F|E}$
\item  $\pr{E|G}$ and  $\pr{F|E}$
\item   $\pr{\brak{E+F}|G}$ and  $\pr{EF|G}$ 
\end{enumerate}
\solution
%\input{./solutions/docq22.tex}

From the given information,
	\begin{align}
	\pr{E} &= \frac{3}{6} = \frac{1}{2} \\
	\pr{F} &= \frac{2}{6} = \frac{1}{3} \\	
	\pr{G} &= \frac{4}{6} = \frac{2}{3} \\	
	\pr{E F} &= \frac{1}{6}\\
	\pr{E G} &= \frac{2}{6}= \frac{1}{3}\\
	\pr{F G} &= \frac{2}{6}= \frac{1}{3}\\
	\pr{E F G} &= \frac{1}{6}
\end{align}
\begin{enumerate}
\item	
\begin{align}
	\pr{E|F} &= \frac{\pr{E F}}{\pr{F}}\\
	\pr{E|F} &= \frac{\frac{1}{6}}{\frac{1}{3}} = \frac{1}{2}\\
	\pr{F|E} &= \frac{\pr{F E}}{\pr{E}}\\
	\pr{F|E} &= \frac{\frac{1}{6}}{\frac{1}{2}} = \frac{1}{3}
	\end{align}

\item 	\begin{align}
	\pr{E|G} &= \frac{\pr{E G}}{\pr{G}}\\
	\pr{E|G} &= \frac{\frac{1}{3}}{\frac{2}{3}} = \frac{1}{2}\\
	\pr{G|E} &= \frac{\pr{G E}}{\pr{G}}\\
	\pr{G|E} &= \frac{\frac{1}{3}}{\frac{1}{2}} = \frac{2}{3}\\
	\end{align}
	
\item	%$\pr{\frac{E\cup F}{G}}$
\begin{multline}
\pr{E+F|G} = \frac{\pr{\cbrak{E+F}G}}{\pr{G}}
\\
 = \frac{\pr{EG + FG}}{\pr{G}}
\\
=\frac{\pr{EG} +\pr{ FG}- \pr{EFG}}{\pr{G}}
\\
 = \frac{3}{4}
\end{multline}
and 
\begin{align}
\pr{EF|G} = 
\frac{ \pr{EFG}}{\pr{G}}
 = \frac{1}{4}
\end{align}

\end{enumerate}


\end{enumerate}

 
\end{document}


