\renewcommand{\thefigure}{\theenumi}
\renewcommand{\thetable}{\theenumi}
%%
%\begin{enumerate}[label=\thesection.\arabic*
%,ref=\thesection.\theenumi]

\subsection{Bernoulli to Gaussian}
\begin{enumerate}[label=\thesubsection.\arabic*
,ref=\thesection.\theenumi]


\item {\em Mean :}  The mean of the bernoulli distribution is 
\begin{align}
\mu = E\brak{X_i}  = \sum_{k=0}^{1}kp_{X_i}(k) = p = \frac{1}{6}
\end{align}
\item {\em Moment:}  The moment of the distribution is defined as
\begin{align}
E\brak{X_i^r}  = \sum_{k=0}^{1}k^rp_{X_i}(k) = p = \frac{1}{6}
\end{align}

%The second moment of the bernoulli distribution is 
%\begin{align}
%E\brak{X_i}  = \sum_{k=0}^{1}kp_{X_i}(k) = p = \frac{1}{6}
%\end{align}
\item {\em Variance :}  The variance of the bernoulli distribution is defined as
\begin{align}
\sigma^2 &= E\brak{X-E\brak{X}}^2  = E\brak{X^2}-E^2\brak{X} 
\\
&=p-p^2 = p\brak{1-p} = \frac{5}{36}
\end{align}
%
The standard deviation 
\begin{align}
\sigma =  \sqrt{p\brak{1-p}}
\end{align}
%
\item {\em The Gaussian Distribution: }  Define
\begin{align}
\label{eq:bern_gauss}
G = \frac{1}{\sqrt{n}}\sum_{k=1}^{n}\frac{X_i-\mu}{\sigma}
\end{align}
%
\item {\em Approximating Binomial Using Gaussian: } From \eqref{eq:bern_gauss}
and \eqref{eq:bern_binom},
%
\begin{align}
X & \approx \sigma\sqrt{n}G + n\mu 
\\
\implies F_X(k) &= \pr{\sigma\sqrt{n}G + n\mu  \le k }
\\
 &= F_G\brak{\frac{k-n\mu}{\sigma\sqrt{n}}} \approx \phi\brak{\frac{k-n\mu}{\sigma\sqrt{n}}} 
\label{eq:bern_gaussian_cdf}
\end{align}
where 
\begin{align}
\phi_{X}(x) = \int^{x}_{-\infty} \frac{1}{\sqrt{2\pi}}e^{-\frac{x^2}{2}}, -\infty < x < \infty
\end{align}
\item The 
probability density function (PDF) 
of $G$ is
%
\begin{align}
p_{G}(x) &= \frac{d}{dx}F_{X}(x)
\\
 &=  \frac{1}{{\sigma\sqrt{n}}}\phi^{\prime}\brak{\frac{k-n\mu}{\sigma\sqrt{n}}} 
\label{eq:bern_gaussian_pdf}
\end{align}
%
For large $n$, $G$ is a continuous distribution with probability density function (PDF)
\begin{align}
p_G(x) =  \frac{1}{\sqrt{2\pi}}\exp\brak{-\frac{x^2}{2}}, -\infty < x < \infty,
\end{align}
%
\item {\em Evaluationg the Probability: }  From \ref{eq:bern_gaussian_cdf}
and \ref{eq:bern_gaussian_pdf},
\begin{align}
\pr{X \le 1 } &= F_{G}(1) = p_G(0)+p_G(1) 
\\
&\approx 
0.41299463887797094
\label{eq:bern_gauss_ans}
\end{align}
which is close to \eqref{eq:bern_binom_ans}.
%
\end{enumerate}
\subsection{Uniform to Gaussian}
\begin{enumerate}[label=\thesubsection.\arabic*
,ref=\thesection.\theenumi]

\item
Generate $10^6$ samples of the random variable
%
\begin{equation}
X = \sum_{i=1}^{12}U_i -6
\end{equation}
%
using a C program, where $U_i, i = 1,2,\dots, 12$ are  a set of independent uniform random variables between 0 and 1
and save in a file called gau.dat
\\
\solution Download the following files and execute the  C program.
\begin{lstlisting}
codes/cdf/exrand.c
codes/cdf/coeffs.h
\end{lstlisting}

%
\item
Load gau.dat in python and plot the empirical CDF of $X$ using the samples in gau.dat. What properties does a CDF have?
\\
\solution The CDF of $X$ is plotted in Fig. \ref{fig:gauss_cdf}
\begin{figure}
\centering
\includegraphics[width=\columnwidth]{./figs/clt/gauss_cdf}
\caption{The CDF of $X$}
\label{fig:gauss_cdf}
\end{figure}


\item
Load gau.dat in python and plot the empirical PDF of $X$ using the samples in gau.dat. The PDF of $X$ is defined as
\begin{align}
p_{X}(x) = \frac{d}{dx}F_{X}(x)
\end{align}
What properties does the PDF have?
\\
\solution The PDF of $X$ is plotted in Fig. \ref{fig:gauss_pdf} using the code below
\begin{lstlisting}
codes/clt/pdf_plot.py
\end{lstlisting}

\begin{figure}
\centering
\includegraphics[width=\columnwidth]{./figs/clt/gauss_pdf}
\caption{The PDF of $X$}
\label{fig:gauss_pdf}
\end{figure}

\item Find the mean and variance of $X$ by writing a C program.
\item Given that 
\begin{align}
p_{X}(x) = \frac{1}{\sqrt{2\pi}}\exp\brak{-\frac{x^2}{2}}, -\infty < x < \infty,
\end{align}
repeat the above exercise theoretically.
%
\item Let $U$ be a uniform random variable between 0 and 1.
%\begin{enumerate}[label=\thesection.\arabic*
%,ref=\thesection.\theenumi]

%
\item
Load the uni.dat file into python and plot the empirical CDF of $U$ using the samples in uni.dat. The CDF is defined as
\begin{align}
F_{U}(x) = \pr{U \le x}
\end{align}
\\
\solution  The following code plots Fig. \ref{fig:uni_cdf}
\begin{lstlisting}
codes/cdf/cdf_plot.py
\end{lstlisting}
\begin{figure}
\centering
\includegraphics[width=\columnwidth]{./figs/cdf/uni_cdf}
\caption{The CDF of $U$}
\label{fig:uni_cdf}
\end{figure}

%\item Generate $10^6$ samples of $U$ using a C program and save into a file called uni.dat .
%\\


%
\item
Find a  theoretical expression for $F_{U}(x)$.

\item
The mean of $U$ is defined as
%
\begin{equation}
E\sbrak{U} = \frac{1}{N}\sum_{i=1}^{N}U_i
\end{equation}
%
and its variance as
%
\begin{equation}
\text{var}\sbrak{U} = E\sbrak{U- E\sbrak{U}}^2 
\end{equation}

Write a C program to  find the mean and variance of $U$. 
\item Verify your result theoretically given that
%
\begin{equation}
E\sbrak{U^k} = \int_{-\infty}^{\infty}x^kdF_{U}(x)
\end{equation}

\end{enumerate}



